\documentclass[itdr]{subfiles}

\begin{document}

\cleartoleftpage

\chapter{Mostri}
\label{ch:mostri}
\index{Mostri}

I mostri sono, per loro stessa natura, diversi dalle persone e dagli animali. Pertanto, essi spesso possiedono capacità speciali al di là dei Punteggi di Abilità. Un dungeon dovrebbe contenere perlopiù mostri unici, ma in questo capitolo sono riportati alcuni esempi.

\vfill
\index{Punti Ferita}
\paragraph{Punti Ferita}
La maggior parte delle creature ha tra 1d6 e 5d6 PF. Ricordate che i Punti Ferita non rappresentano semplicemente la capacità di assorbire danni fisici, ma anche l'astuzia e l'abilità del mostro di evitare di farsi male.

\vfill
\paragraph{Uccisione dei Mostri}
A parte ove segnalato, i mostri sono trattati esattamente come i personaggi.

\vfill
\index{Magia!mostri}
\paragraph{Magia}
Sebbene alcuni mostri potrebbero fare uso di Incantesimi nello stesso modo dei Mistici, qualcheduno è capace di usare Incantesimi senza un Tomo o un Focus. I mostri possono non seguire quelle regole.

\vfill
\index{Armature}
\paragraph{Armature}
Usate l'armatura dei personaggi come guida per la rappresentazione di mostri dalla pelle dura o di quelli talmente grossi da essere in grado di scrollarsi di dosso la maggior parte delle armi.

\vfill
\index{Danno!mostri}
\paragraph{Danno}
Se non viene menzionato nulla, gran parte dei mostri infligge d6 Danno. Alcuni hanno un dado Danno più grande o anche un dado bonus del Danno.

\vfill
\index{Perdita di Punti Abilità}

\index{Morte}
\paragraph{Perdita di Punti Abilità e Attacchi Mortali}
Creature particolarmente letali potrebbero ridurre il Punteggio di Abilità del bersaglio qualora dovesse fallire un Tiro Salvezza, provocando un destino orribile quando il punteggio viene azzerato.

\vfill
\index{Punteggi di Abilità}
\paragraph{Una Nota sui Punteggi di Abilità}
Quando si assegnano i Punteggi di Abilità, 20 dovrebbe di norma essere considerato il massimo. Si potrebbe pensare che un mostro enorme dovrebbe avere una FOR di 30 o più, ma va considerato che le creature di grandi dimensioni potrebbero non combattere così bene. Le loro dimensioni dovrebbero invece essere rappresentate dalla capacità di infliggere più Danno e dal possedere un punteggio di Armatura più alto.

\vfill
\break

\section{Conversione dei Mostri}
\index{Conversione dei Mostri}

\subsection*{5\ap{a} Edizione}

\subparagraph{PF:} 1pf per DV. Massimo 30.\\Se non c'è l'indicazione dei DV, DV=PF/5 (arrotondando per difetto).
\subparagraph{Armatura:} va aumentata di 1 per armature rinomate, estrema resilienza e per ciascuna categoria di taglia sopra quella Media.
\subparagraph{Punteggi delle Abilità/Caratteristiche:} trasferibili direttamente, usando il CAR per la VOL. Massimo 20.
\subparagraph{Attacchi:} iniziano con un d6. La taglia del dado va aumentata di uno per ciascuna categoria di taglia sopra quella Media e una volta in più se è impugnata un'arma pesante. Niente attacchi multipli.
\subparagraph{Vulnerabilità / Resistenze:} da rimpiazzare rispettivamente con Potenziamento / Compromissione.

\vfill

\subparagraph{Altre Edizioni:} come la 5\ap{a} edizione, eccetto:
\subsection*{4\ap{a} Edizione}
\subparagraph{PF:} 1pf per Livello. $\times$3 per le creature Solitarie, +1pf per creature Piccole o più grandi.
\subparagraph{Punteggi delle Abilità/Caratteristiche:} come la 5\ap{a} edizione, eccetto:
\begin{itemize}
	\item --4~FOR per Umanoidi e Mostruosità
	\item --2~FOR per creature Non Morte
	\item --4~DES per creature di taglia Grande o maggiore
	\item --2~DES per creature Umanoidi o Non Morte di taglia Media o inferiore
	\item --2~VOL per le Mostruosità
\end{itemize}

\vfill
\subsection*{3\ap{a} Edizione e 3.5}
\subparagraph{DV:} 1pf per DV. +1pf per creature Piccole o Medie e +2pf per creature di taglia Grande o maggiore, Melme escluse.
\subparagraph{Punteggi delle Abilità/Caratteristiche:} se la FOR non è specificata, inferiore a 10.

\vfill
\subsection*{Edizioni Original, Basic e Advanced}
\subparagraph{PF:} 1pf per DV. +1pf per creature Piccole o Medie e per Melme di taglia Grande o maggiore; +2pf per creature di taglia Grande o maggiore.

\subparagraph{Morale:} si continuano a usare 2d6 (Original e Basic) o 2d10~(Advanced) oppure si convertono a d20 (VOL):

\begin{dtable}[CCC|CCC]
	\textbf{2d6} & \textbf{2d10} & \textbf{d20} & \textbf{2d6} & \textbf{2d10} & \textbf{d20} \\
	2 & 2--3 & 1	& 7 & 11--12 & 11--13 \\
	3 & 4--5 & 2	& 8 & 13--14 & 14--16 \\
	4 & 6--7 & 3--4	& 9 & 15	 & 17 \\
	5 & 8	 & 5--6	& 10& 16--17 & 18 \\
	6 & 9--10& 7--9 & 11& 18--19 & 19 \\
\end{dtable}
\vfill
\break

\section{Idee per la Creazione di Mostri}

\paragraph{Aspetto e Comportamento}
Cambiate l'aspetto esteriore e il comportamento del mostro esistente. Altre possibilità potrebbero essere il cambiare le sue dimensioni o il combinare un paio di mostri in un'unica creatura.

\vfill
\paragraph{Tratti dei Personaggi}
\index{Personaggi-Non-Giocanti}
\index{PNG|see {Personaggi-Non-Giocanti}}
Applicate ai personaggi-non-giocanti (PNG) e ai mostri i Tratti dal \textbf{\fullref{ch:personaggi}}, in special modo ai ``boss''.

\vfill
\paragraph{Effetti da Danno Critico}
\index{Danno!critico}
Con un Tiro Salvezza fallito, il bersaglio di un mostro subisce qualche effetto negativo aggiuntivo: infermità, veleno, Perdita di Punti Abilità o finanche la morte. Va deciso se in questi casi al bersaglio è concesso un Tiro Salvezza per evitarlo.

\vfill
\paragraph{Coppie}
Un tipo di mostri Potenzia gli attacchi dell'altro, fornisce protezione o qualche altro beneficio.

\vfill
\paragraph{Rafforzamenti}
Un mostro riceve un rafforzamento, un nuovo attacco o cambia tattica quando esaurisce i suoi PF, si salva da un Danno Critico per la prima volta, prende Danno da una fonte specifica, ecc.

\vfill
\paragraph{Attacchi e Capacità Speciali}
Al posto del suo attacco predefinito, un mostro può usarne uno speciale, come una capacità simile a un Incantesimo o altro effetto non comune. Alcune di queste capacità potrebbero essere ``passive'' (sempre attive).

\vfill
\paragraph{Tattiche e Armi}
I mostri potrebbero ricorrere a tattiche di combattimento inaspettate, specialmente quando combattono in gruppi. Se un mostro è armato, si può cambiare la sua arma in qualcosa di insolito o passare da arma da mischia ad arma a distanza o viceversa.

\vfill
\paragraph{Vulnerabilità, Resistenze e Immunità}
\index{Attacchi!potenziati}
\index{Attacchi!compromessi}
Specifici attacchi contro il mostro sono Potenziati, Compromessi o non funzionano affatto.

\vfill
\begin{dbox}
	Vedi l'\textbf{\customref{ch:appendice_b}{Appendice B: Bestiario}} per esempi di mostri e ulteriore ispirazione.
\end{dbox}

\vfill
\break

\section{Esempi di Capacità dei Mostri}

\paragraph{Afferrare}
Se un bersaglio fallisce un \save{DES}, è Frenato finché non supera un \save{FOR o DES} nei turni successivi. I mostri non possono attaccare con gli arti con cui sono impegnati ad afferrare, ma quelli con particolare forza fisica potrebbero invece infliggere Danno al bersaglio afferrato.

\vfill
\paragraph{Arti Extra}
Un mostro ha più dadi del Danno (prende comunque quello più alto per un singolo bersaglio). Alcuni mostri possono anche attaccare più opponenti, distribuendo i dadi del Danno tra questi attacchi.

\vfill
\paragraph{Assorbimento}
Quando un mostro prende Danno da un certa fonte (di solito, una elementale), ripristina invece di un pari ammontare i suoi PF (o anche la FOR).

\vfill
\paragraph{Carica}
Un mostro riduce rapidamente la distanza dal bersaglio. Quest'ultimo deve superare un \save{DES} o subisce un Danno maggiorato e/o altri effetti.

\vfill
\paragraph{Debolezza}
Quando un mostro subisce Danno da una fonte associata alla sua debolezza (anche se questo Danno non è il più alto del turno in corso), perde alcuni dei suoi poteri, viene Stordito, ecc. Di solito, tali effetti durano per il turno successivo del mostro.

\vfill
\paragraph{Indomabile}
Una volta a Riposo, quando prende Danno Critico, un mostro continua a combattere come se avesse superato il relativo Tiro Salvezza. Alcuni mostri non viventi o non morti potrebbero ingnorare del tutto il Danno Critico.

\vfill
\paragraph{Ingoiare}
Il bersaglio deve superare un \save{DES} o viene completamente ingoiato, subendo una Perdita di Punti Abilità (FOR, DES o ambedue) a ciascun turno seguente. Se il mostro subisce Danno Critico, deve superare un ulteriore \save{FOR} o rigurgita tutte le creature ingoiate.

\vfill
\paragraph{Volatile}
Quando un mostro subisce Danno Critico, esplode infliggendo Danno da Scoppio a chiunque gli sia vicino.

\vfill

\end{document}
