\documentclass[itdr]{subfiles}

\begin{document}

\addtocontents{toc}{\protect\newpage}
\chapterx{Appendice A: Regole Aggiuntive e Alternative}
\label{ch:appendice_a}

``\title'' è da intendersi come un gioco dal regolamento leggero: tenetelo a mente se deciderete di usare una qualsiasi delle regole presenti in questa appendice.
%\vspace{-0.4ex}
\vfill

\section{Personaggi}
\index{Personaggi}
\index{Personaggi-Giocanti}
\index{PG|see {Personaggi-Giocanti}}

\subsection{Personaggi Equilibrati}
Invece di usare il Tiro Extra per il denaro iniziale, ignoratelo. Il denaro di partenza è 21 meno la media dei Punteggi di Abilità (arrotondando per eccesso).

\subsection{Personaggi Epici}
Se volete che i personaggi-giocanti siano più potenti, tirate 2d6~+~6 per i Punteggi di Abilità e d4~+~2 per i PF.

\subsection{Personaggi Comuni}
Se volete che i personaggi-giocanti siano persone comuni, tirate 2d8~+~1 per i Punteggi di Abilità e d6 per i PF. \mbox{Poi, niente Tratti, scegliete solo un Background.}

Quando dovrete avanzare un personaggio simile a Novizio, scegliete un Tratto, tirate di nuovo per i PF tenendo il risultato migliore e tirate d20 per ciascun Punteggio di Abilità: se il tiro è più alto del Pungeggio di Abilità, allora questo sale di 1 (fino a 18).

\subsection{La Fortuna Aiuta l'Audace}
Quando create il personaggio o avanzate a un nuovo Livello d'Esperienza, invece di scegliere un nuovo Tratto, fate un tiro per prenderne uno a caso e fate lo stesso per eventuali Incantesimi random, \mbox{Competenze} e Doni\safepageref{ (see pagina }{caratteristiche_casuali}{)}, per ottenere in cambio uno dei seguenti benefici:
\begin{itemize}
	\item Fare un tiro in più per i PF e prendere il risultato migliore.
	\item Aumentare un Punteggio di Abilità di 1 (fino a 20).
\end{itemize}

\vfill

\section{Contese}
\index{Contese}
In una disputa per cui un semplice Tiro Salvezza non sarebbe sufficiente, entrambe le parti fanno un TS: se una delle parti ha successo, vince. Se il TS riesce a entrambe, vince chi ha fatto il tiro più basso. In caso di pareggio, vince chi ha il Punteggio di Abilità più alto.

Se sono coinvolte delle armi, l'attacco potrebbe essere sottratto dal tiro o aggiunto a quello dell'opponente.

\vfill

\section{Tiri Salvezza di Gruppo}
\index{Tiri Salvezza!gruppo}
\index{Tiri Salvezza di Gruppo|see {Tiri Salvezza, gruppo}}
Quando l'intero gruppo agisce come un tutt'uno, potrebbe essere fatto un Tiro Salvezza di Gruppo. Se più della metà dei personaggi supera il proprio Tiro Salvezza, il tiro riesce. Il tentativo di governare un'imbarcazione durante una tempesta potrebbe richiedere un Tiro Salvezza di Gruppo di~FOR e superare furtivamente le guardie un Tiro Salvezza di Gruppo di~DES.

\vfill

\begin{comment}
\section{Gods, Religion, and Disciples}
\index{Religion}
\index{Disciple}
\index{Creed}

The nature of divine presence is highly dependent on a specific setting and thus is left to your discretion. Some worlds could be completely devoid of divine influence (though local cults might still have supernatural powers from some other source), while dwellers of other worlds can regularly observe their gods' interventions in the deals of mortals.

\subparagraph{Disciple} Class and its Creeds from the \textbf{\customref{ch:appendix_c}{Appendix C: Class-ic Edition}} could be used as a Feature to represent the most devoted adepts of cryptic cults. Unlike other Features, this one has a prerequisite of the character being a worshipper of the relevant set of teachings. When obtaining a new Experience Level, follow standard rules. Additionally, from Expert onwards, Disciples gain d4 (up to their WIL~/~2, rounded down) Followers (3hp, Simple Weapon) each time they visit a friendly settlement  and are responsible for their food, shelter, equipment, etc.
\end{comment}

\vfill

\section{Modalità Estrema}
Per aumentare la difficoltà, usate le regole seguenti:
\begin{itemize}
	\item I Mistici usano la regola della \textbf{Selezione Casuale degli Incantesimi}.
	\item Quando i Mistici falliscono il loro TS per Arcanocombustione Critica, subiscono un \textbf{Contrattempo Magico}.
	\item I personaggi che prendono \textbf{Danno Critico} hanno bisogno che un personaggio alleato spenda la propria azione per occuparsi della ferita o Perderà d6~FOR a ciascun turno successivo.
	\item Usate la regola delle \textbf{Lesioni}. I tiri di \textbf{Arto rotto} impongono invece la perdita dell'arto. I tiri di \textbf{Ferita gravissima} \mbox{cambiano} in morte istantanea.
\end{itemize}

\vfill

\section{Lesioni}
\index{Lesioni}
\index{Danno!critico}

Se fallisci un Tiro Salvezza per Danno Critico, tira per una lesione.
Gli effetti di una lesione potrebbero essere risolti tramite Guarigione.

\begin{dtable}[cL]
	\textbf{d20} & \textbf{Lesione} \\
	1--4	& \textbf{Ematoma.} Nulla di serio.\\
	5--7	& \textbf{Cicatrice.} Questa lascerà un segno.\\
	8--9	& \textbf{Commozione cerebrale.} Svant. ai \saves{VOL}.\\
	10--11	& \textbf{Costola incrinata.} Svant. ai \saves{DES}.\\
	12--13	& \textbf{Muscolo strappato.} Svant. ai \saves{FOR}.\\
	14--15	& \textbf{Attrezzatura rotta.} Ordine: scudo $\rightarrow$ armatura $\rightarrow$ arma. Ripararla costa la metà del suo prezzo.\\
	16		& \textbf{Cranio fratturato.} Perdita di d6 VOL.\\
	17		& \textbf{Costole rotte.} Perdita d6 DES.\\
	18		& \textbf{Emorragia interna.} Perdita di d6 FOR.\\
	19		& \textbf{Arto rotto.} Tira per l'arto. d4: \mbox{1--2 =} braccio sinistro/destro (non puoi più usarlo), \mbox{3--4 = gamba} sinistra/destra (non puoi più correre, saltare, ecc. L'arto viene perduto se subisce un secondo colpo).\\
	20		& \textbf{Ferita gravissima.} Il prossimo Tiro Salvezza per Danno Critico che si fallisce porterà alla morte.\\
\end{dtable}

\vfill

\section{Luce}
\index{Luce}

Torce, lanterne e fuochi da campo illuminano in un raggio di \mbox{30 piedi}. Grandi falò potrebbero proiettare luce al doppio della distanza. Candele e simili illuminano solo in un raggio di 10 piedi e pertanto non comunemente utilizzate da chi va in Avventura.

Foschia, fumo e simili riducono il raggio della metà.

\subparagraph{Torcia:} dura per circa un'ora. Se usata come arma, fa d4 Danno da Fuoco, ma potrebbe spegnersi.

\subparagraph{Lanterna:} dura circa quattro ore, la luce che proietta può essere smorzata quando si vuole ed è ricaricabile con olio da lampada.

\vfill

\section{Spese di Sostentamento}
\index{Spese di Sostentamento}
\index{Stile di Vita|see {Spese di Sostentamento}}

\equip{Squallido}{d4s/mese}:
Perdi d4 Punti Abilità per ciascun Punteggio di Abilità, la tua reputazione ne risente.

\equip{Adeguato}{10$\times$d4s/mese}:
recuperi d4 Punti Abilità Persi per ciascun Punteggio di Abilità.

\equip{Opulento}{d4f/mese}:
sani qualsiasi Perdita di Punti Abilità e problematica di salute non magica, la tua reputazione cresce.

Se possiedi compagni animali, aggiungi la metà per ciascuno di essi.

Dimezza le spese se vivi a casa tua.

\vfill

\section{Capacità di Carico}
\index{Capacità di Carico}

I personaggi possono \textbf{sollevare} un quantitativo massimo di carico pari alla loro FOR elevata al quadrato in libbre (lb). Si può \textbf{trasportare} senza ostacoli, alla propria velocità, la \textbf{Metà} di questo carico. Il \textbf{Doppio} può essere \textbf{trascinato} sul terreno.

\begin{dtable}[llLl]
	\textbf{FOR} & \textbf{Trasp. (\sfrac{1}{2}), lb} & \textbf{Sollev., lb} & \textbf{Trasc. ($\times$2), lb} \\
	1	&	\sfrac{1}{2}		&	1		&	2		\\
	2	&	2		&	4		&	8		\\
	3	&	4\sfrac{1}{2}		&	9		&	18		\\
	4	&	8		&	16		&	32		\\
	5	&	12\sfrac{1}{2}		&	25		&	50		\\
	6	&	18		&	36		&	72		\\
	7	&	24\sfrac{1}{2}		&	49		&	98		\\
	8	&	32		&	64		&	128		\\
	9	&	40\sfrac{1}{2}		&	81		&	162		\\
	10	&	50		&	100		&	200		\\
	11	&	60\sfrac{1}{2}		&	121		&	242		\\
	12	&	72		&	144		&	288		\\
	13	&	84\sfrac{1}{2}		&	169		&	338		\\
	14	&	98		&	196		&	392		\\
	15	&	112\sfrac{1}{2}		&	225		&	450		\\
	16	&	128		&	256		&	512		\\
	17	&	144\sfrac{1}{2}		&	289		&	578		\\
	18	&	162		&	324		&	648		\\
	19	&	180\sfrac{1}{2}		&	361		&	722		\\
	20	&	200		&	400		&	800		\\
\end{dtable}
{\em (Trasp. = trasportato, Sollev. = sollevato, Trasc. = trascinato)}

1~lb è uguale al perso di 100 fiorini d'oro, 1000 scellini d'argento o 1000 pennies di rame.

\subsection{Ingombro}
\index{Ingombro}
Al di là di una penalità alla velocità, un carico pesante \textbf{riduce i PF a 0}. La stessa riduzione dei PF avviene quando si trasportano \textbf{più di tre} oggetti scomodi da spostare. Gli oggetti sono considerati scomodi da spostare se richiedono entrambe le mani per poter essere trasportati o sono comunque poco maneggevoli, come armi a due mani, armature, un Tomo del Mistico, un vaso di polvere nera, ecc.


\vfill
\break


\section{Follia}
\index{Follia}
\index{Sanità|see {Follia}}
\index{Insania|see {Follia}}
Se la vostra partita è incentrata particolarmente su di un tema horror, potreste volere tracciare la sanità dei personaggio.

Ogni volta che il personaggio viene esposto a una fonte di terrore soprannaturale, deve superare un \save{VOL} o ottiene un Livello di Follia.

Una bella dormita di una notte riduce di 1 il Livello di Follia.

Quando il Livello di Follia supera il Livello d'Esperienza del personaggio, quest'ultimo impazzisce: tirate per l'Effetto Immediato e per quello Prolungato. Gli Effetti Prolungati \mbox{richiedono} un Trattamento Curativo per poter essere eliminati.

\begin{dtable}[cL]
	\textbf{d20} & \textbf{Effetto Immediato} \\
	1--4 & \textbf{Tremore}. \\
	5--7 & \textbf{Urli} a gran voce, facendo parecchio rumore. \\
	8--10 & \textbf{Ti dimeni}, attaccando al tuo prossimo turno un bersaglio vicino a caso. \\
	11--13 & \textbf{Vai in panico} e corri via. 2 possibilità su 6 di mollare la tua arma mentre lo fai. \\
	14--15 & \textbf{Frenesia}. Spendi i tuoi turni ad attaccare un bersaglio vicino a caso. Dopo averne attaccato uno alleato, puoi riprenderti superando un \save{VOL}. \\
	16--17 & \textbf{Cecità} finché non Riposi. \\
	18--19 & \textbf{Paralisi} finché non arriva un Danno o chicchesia non usi la propria azione per scrollartela di dosso. \\
	20 & \textbf{Svieni}. Per riprendere conoscenza, hai bisogno dell'assistenza di un soggetto alleato e di un Riposo. \\
\end{dtable}

\begin{dtable}[cL]
	\textbf{d20} & \textbf{Effetto Prolungato} \\
	1--4 & \textbf{Incubi}. \\
	5--7 & \textbf{Allucinazioni}. A discrezione dell'Arbitro. \\
	8--10 & \textbf{Mutismo}. Perdi la capacità di parlare. \\
	11--13 & \textbf{Fobia}. Gli attacchi contro la causa della fobia sono Compromessi. \\
	14--15 & \textbf{Paranoia}. Svantaggio ai \saves{VOL}. \\
	16--17 & \textbf{Vertigini}. Svantaggio ai \saves{DES}. \\
	18--19 & \textbf{Debolezza}. Svantaggio ai \saves{FOR}. \\
	20 & \textbf{Torpore}. Non puoi eseguire azioni. \\
\end{dtable}

\paragraph{Effetti Residuali}
Alcune esperienze particolarmente scioccanti potrebbere lasciare un segno permanente sulla psiche del personaggio, in genere come forma leggera di qualche Effetto Prolungato, un'ossessione, un comportamento compulsivo, ecc.

\vfill
\break

\section{Contrattempi Magici}
\label{sec:contrattempi_magici}
\index{Magia!contrattempi}
\index{Arcanocombustione}

Quando i Mistici falliscono il proprio Tiro Salvezza per Arcanocombustione Critica, subiscono un Contrattempo. Un Contrattempo potrebbe verificarsi anche a causa di altre \mbox{interazioni} pericolose con la magia (uso improprio di un dispositivo magico, lancio di Incantesimi in una zona di anti-magia, distruzione di un oggetto magico, ecc.).

\begin{dtable}[cL]
	\textbf{d100} & \textbf{Contrattempo} \\
	1--4	&	Emetti un forte odore per un giorno. \mbox{d4: 1 =~menta}, 2 = ~aglio, 3 =~aceto, 4 =~zolfo.	\\
	5--8	&	Il colore dei tuoi vestiti cambia a caso.	\\
	9--12	&	I tuoi vestiti crescono di una taglia. Ottieni Svantaggio ai \saves{DES} finché non sono riparati.	\\
	13--16	&	Il colore dei tuoi occhi cambia a caso.	\\
	17--20	&	I tuoi occhi emettono luce intensa per un giorno.	\\
	21--24	&	Il colore dei tuoi capelli cambia in un colore a caso (i nuovi capelli crescono normali).	\\
	25--28	&	I tuoi capelli cadono.	\\
	29--32	&	I tuoi capelli crescono come se fosse passato un anno.	\\
	33--36	&	La tua pelle prende una tonalità satura di un colore a caso per d12 mesi.	\\
	37--40	&	La tua pelle si ricopre di una crescita a caso per d12 mesi. d4: 1 =~pelliccia, 2 =~scaglie, \mbox{3 =~piume}, 4 =~spine.	\\
	41--43	&	Svanisci per un minuto.	\\
	44--46	&	Prendi la condizione Stordito finché non Riposi.	\\
	47--49	&	Prendi la condizione Svenuto finché non Riposi.	\\
	50--52	&	Prendi la condizione Invisibile per un'ora o finché non attacchi o non lanci un Incantesimo.	\\
	53--55	&	Le tue orecchie diventano a punta e pelose.	\\
	56--58	&	Non puoi più udire suoni finché non Riposi.	\\
	59--61	&	La tua voce ha un volume molto alto finché non Riposi.	\\
	62--64	&	Non puoi più parlare finché non Riposi.	\\
	65--67	&	Vedi le cose Invisibili per un'ora.	\\
	68--70	&	Prendi la condizione Accecato finché non Riposi.	\\
	71--72	&	Una nuvola di fumo ti ammanta.	\\
	73--74	&	I tuoi PF scendono a 0.	\\
	75--76	&	I tuoi PF sono ripristinati.	\\
	77--78	&	Raddoppi la tua taglia per un'ora. Ottieni Vantaggio ai \saves{FOR} e Incrementi il dado del Danno delle tue armi.	\\
	79--80	&	Dimezzi la tua taglia per un'ora. Ottieni Svantaggio ai \saves{FOR} e Riduci il dado del Danno delle tue armi.	\\
	81--82	&	La tua arma principale si restringe a un sedicesimo della sua taglia per un'ora.	\\
	83--84	&	La tua lingua diventa biforcuta.	\\
	85--86	&	I tuoi canini diventano lunghi e affilati.	\\
\end{dtable}
\begin{dtable}[cL]
\textbf{d100} & \textbf{Contrattempo} \\
	87--88	&	Ti cresce una coda.	\\
	89--90	&	Ti crescono delle corna.	\\
	91	&	Ti crescono delle branchie.	\\
	92	&	I tuoi piedi si trasformano in zoccoli.	\\
	93	&	Le tue unghie crescono fino a diventare artigli affilati (d6 come dado bonus per il Danno senz'armi).	\\
	94	&	La tua pelle diventa molto dura. Ottieni Armatura~1 quando non indossi un'armatura.	\\
	95	&	Uno dei tuoi Punteggi di Abilità aumenta di un punto (fino a 20). d6: 1--2 =~FOR, 3--4 =~DES, \mbox{5--6 =~VOL.}	\\
	96	&	Uno dei tuoi Punteggi di Abilità diminuisce di un punto (fino a 3). d6: 1--2 =~FOR, \mbox{3--4 =~DES,} 5--6 =~VOL.	\\
	97	&	Ti cresce una parte del corpo a caso.	\\
	98	&	Perdi una parte del corpo a caso.	\\
	99	&	I tuoi vestiti vengono avvolti dalle fiamme. Prendi d6 Danno ora e d6 alla fine del tuo prossimo turno, a meno che non vengano spente.	\\
	100	&	Ti trasformi in pietra.	\\
\end{dtable}

\vfill

\begin{dtable}[cLcL]
	\textbf{d12} & \textbf{Colore} & \textbf{d12} & \textbf{Colore} \\
	1 & bianco neve		& 7	& giallo limone \\
	2 & grigio cenere		& 8	& verde malachite \\
	3 & nero pece		& 9 & blu cielo \\
	4 & rosso cremisi		& 10& blu oltremare \\
	5 & marrone castagna	& 11& viola lavanda \\
	6 & arancione zucca	& 12& magenta orchidea \\
\end{dtable}

\vfill

\begin{dtable}[cLcLcL]
	\textbf{d12} & \textbf{Parte del Corpo} & \textbf{d12} & \textbf{Parte del Corpo} \\
	1		& dente	& 7--9 	 & dito del piede	\\
	2--4	& dito della mano& 10--11 & piede	\\
	5--6	& braccio	& 12	 & occhio	\\
\end{dtable}
\begin{comment}
% pre-1.3 version
\begin{dtable}[cLcL]
	\textbf{d20} & \textbf{Part} & \textbf{d20} & \textbf{Part} \\
	1		& tooth	& 11--14 & toe	\\
	2--5	& finger& 15--17 & foot	\\
	6--8	& hand	& 18--19 & leg	\\
	9--10	& arm	& 20	 & eye	\\
\end{dtable}
\end{comment}

\vfill

\section{Fabbricare Equipaggiamento Magico}
\label{sec:fabbricare_equipaggiamento_magico}
\index{Magia!equipaggiamento}
\index{Focus}
\index{Pergamene}

L'equipaggiamento magico basilare può essere creato da un Mistico spendendo l'ammontare richiesto di fondi e di tempo.

\subparagraph{Focus}: 10s in risorse, d4 giorni, un oggetto idoneo.

\subparagraph{Pergamena}: 20s $\times$ Cerchio in risorse, d4 giorni $\times$ Cerchio. Riesce X su 6 volte, X = 1 + Livello da \mbox{Mistico -- Cerchio}, le risorse sono perse in ambo i casi (ideare un nuovo Incantesimo, se l'Arbitro lo consente, costa risorse e richiede giorni almeno $\times$10 volte tanto e prevede l'uso di alcuni rari ingredienti).

\vfill

\section{Esperienza dei Compagni Animali}
\index{Compagni Animali}
\index{Livelli d'Esperienza!compagni animali}

Se volete consentire che i compagni animali accumulino esperienza, fatelo \textbf{una volta sola}, quando un compagno animale supera tre Avventure. Usate le stesse regole per l'aumento di Punteggi di Abilità e PF usate per i personaggi.

\break

\section{Razioni}
\index{Razioni}

Quando si viaggia per mare o in terre inospitali, può essere importante conoscere l'ammontare e il peso delle razioni richieste per il viaggio.

\begin{dtable}[Lllll]
	\textbf{Razioni \mbox{Quotidiane}} & \textbf{Costo} & \textbf{Cibo} & \multicolumn{2}{l}{\textbf{Acqua}} \\
	Essere umano	& 5p	& 2~lb	& \sfrac{1}{2}~gal	& (4~lb) \\
	Cavallo	& 1p	& 20~lb	& 5~gal	& (40~lb) \\
	Elefante& 1s	& 200~lb& 50~gal& (400~lb) \\
\end{dtable}

Un giorno senza abbastanza acqua o una settimana senza abbastanza cibo impongono una Perdita di d4 FOR.

\vfill

\section{Risorse}
\index{Risorse}

\begin{dtable}[cLcL]
	\textbf{Quantità} & \textbf{Descrizione} & \textbf{Media} & \textbf{Prezzo} \\
	1 & in \mbox{esaurimento}	& 1		& $\times$~1 \\
	2 & bassa			& 2		& $\times$~d6 \\
	3 & sufficiente		& 4		& $\times$~2d6 \\
	4 & abbondante		& 7		& $\times$~3d6 \\
	5 & in eccedenza		& 13	& $\times$~4d6 \\
\end{dtable}

Ogni volta che spendi una risorsa (o, per le munizioni, dopo il combattimento), tira un d6: sa fai più della Quantità, riducila di uno. A zero, la risorsa è terminata.

Se raccogli una risorsa, tira un d6: sa fai più della Quantità, aumentala di 1 (fino a 5).

Quando acquisti le risorse, aumentane la Quantità di 1 (fino a 5), pagandone il prezzo moltiplicato per la tua attuale Quantità~$\times$~d6.

\vfill

\section{Vendita}
\index{Vendita}
\index{Baratto}

La probabilità di trovare chi compra un oggetto costoso è \mbox{X su 6}, a seconda dell'insediamento e del costo dell'oggetto. Puoi ripetere la ricerca nello stesso insediamento soltanto dopo d6 mesi.

\begin{dtable}[lLLLLLl]
	\textbf{Oro:} &	\textbf{1+} & \textbf{10+} & \textbf{100+} & \textbf{1k+} & \textbf{10k+} & \textbf{100k+} \\
	Villaggio	& 2	& 1		& ---	& ---	& ---	& --- \\
	Cittadina	& 4 & 3		& 2		& 1		& ---	& --- \\
	Città	& 6	& 5		& 4		& 3		& 2		& 1 \\
\end{dtable}

Dopo aver trovato chi vuole comprare l'oggetto, fa' un Tiro Salvezza di VOL. In caso di fallimento, vendi per un \sfrac{1}{4} del prezzo. Se fai meno del tuo Punteggio di VOL di 10 o più, vendi per il prezzo pieno, altrimenti vendi per un \sfrac{1}{2} del prezzo. La probabilità di un baratto invece di uno scambio monetario è (6--X) su 6.

\subparagraph{Vendita di Oggetti Magici}: gli oggetti magici avranno una maggiore probabilità di essere barattati, mentre il tiro per la ricerca e il \save{VOL} vanno fatti con Svantaggio. Il prezzo delle Pergamene è d10s~$\times$~Cerchio, consumabili = d10$\times$10s~$\times$~Cerchio, bacchette e \mbox{verghe = 10f}~$\times$~Cerchio, altri oggetti = valutati \mbox{caso per caso}.

\vfill
\break

\section{Strutture e Assedi}
\label{sec:strutture_e_assedi}
\index{Strutture}
\index{Assedi}

\subsection{Costruzione}
\index{Costruzione}
\index{Proprietà}
\index{Mura}

\begin{dtable}[lLll]
	\multicolumn{2}{L}{\textbf{Struttura}} & \textbf{Legno} & \textbf{Pietra} \\
	\multicolumn{2}{L}{Edificio~(1 piano), P=100~piedi}		& 1f	& 5f \\
	\multicolumn{2}{L}{Guardiola, P=200~piedi}				& 10f	& 50f \\
	\multicolumn{2}{L}{Mura, 100~piedi}					& 1f	& 5f \\
	\multicolumn{2}{L}{Ponte, 100 piedi}					& 1f	& 5f \\
	\multicolumn{2}{L}{Rocca piccola, P=200~piedi}			& 20f	& 100f \\
	\multicolumn{2}{L}{Rocca grande, P=400~piedi}				& ---	& 300f \\
	\multicolumn{2}{L}{Torre piccola, P=100~piedi}			& 5f	& 25f \\
	\multicolumn{2}{L}{Torre grande, P=200~piedi}			& 10f	& 50f \\
	\hline
	Dungeon, 10~piedi cubi & \multicolumn{3}{L}{20s (terra), 1f (roccia)} \\
	Fossato, 100~piedi	& \multicolumn{3}{L}{1f (terra), 5f (roccia)} \\
	Strada, 1~miglio	& \multicolumn{3}{L}{5f su terreno Sgombro, 10f su Accidentato, 20f su Difficile} \\
\end{dtable}
{\em (P --- perimetro esterno della costruzione)}

\subparagraph{Squadra di Costruzione:} una squadra (quattro dozzine di persone dirette da un mastro), pagata 50s alla settimana, costruisce 5f di costi per strutture alla settimana, 1f per opere in pietra. A una singola struttura possono lavorare in simultanea fino a 5 squadre. Velocità e costo potrebbero variare in virtù di fattori esterni.

\subparagraph{Macchine d'Assedio:} piazzabili su guardiole (1), torri grandi (1), rocche piccole (2) e rocche grandi (4).

\vfill

\subsection{Macchine d'Assedio}
\index{Macchine d'Assedio}
\index{Danno!macchine d'assedio}

Richiedono una squadra di tre componenti e un turno intero per la ricarica. Una squadra ridotta le ricaricherà in due o tre turni.

\begin{dtable}[LlLL]
	\textbf{Macchina} & \textbf{Costo} & \textbf{Danno} & \textbf{Munizione} \\
	Ballista 	& 1f 	& d12 		& 10s a dardo \\
				& 	 	& d10 		& 5s a palla \\
	\rowcolor{dColor1}\multirow{-2}{*}{Catapulta} & \multirow{-2}{*}{1f}	& d10 Scoppio & 20s a bomba \\
	\rowcolor{dColor2}Cannone & 2f	& d12 Scoppio & 25s a sparo \\
\end{dtable}

Il peso di una macchina d'assedio è di circa 1 t e richiede un'animale da tiro per il trasporto via terra.

\vfill

\subsection{Danno Strutturale}
\index{Danno!strutturale}

La gamma di Armatura rappresenta lo spessore del materiale.

Oggetti più grossi e più grandi di norma ignorano il Danno da fonti diverse da macchine d'assedio o simili.

\begin{dtable}[lcL|lc]
	\textbf{Taglia} & \textbf{PF} & \textbf{Esempio} & \textbf{Materiale} & \textbf{Armatura} \\
	piccola	& 2--4	& scrigno	& ghiaccio 	& 2--4	\\
	media	& 4--8	& carrozza	& legno	& 4--6	\\
	grande	& 6--12	& mura	& pietra	& 6--8	\\
	enorme	& 8--16	& nave	& metallo	& 8--10	\\
\end{dtable}

Per esempio, una nave di legno di dimensioni modeste avrà 8pf e Armatura~5 (legno di medio spessore).

\vfill
\break

\section{Viaggio}
\index{Viaggio}

Si può viaggiare per \textbf{8 ore/giorno} prima di doversi fermare per la notte.

\subparagraph{Griglia}: una griglia di caselle da \travelunit{1} miglia semplifica il calcolo delle distanze.

\index{Terreno}
\begin{dtable}[lLcc]
\textbf{Terreno} & \textbf{Esempio} & \textbf{Miglia} & \textbf{Griglia} \\
Sgombro		& prateria, pianure 		& \travelunit{4} & 4 \\
Accidentato		& deserto, foresta, colline		& \travelunit{3} & 3 \\
Difficile	& \mbox{giungla, montagne,} palude	& \travelunit{2} & 2 \\
\end{dtable}

Per velocizzare i calcoli, scegliete un \mbox{terreno} dominante per un giorno (o per una mezza giornata) di viaggio e applicatelo all'intera durata.

\index{Sfinimento}
\subparagraph{Prova di Sfinimento:} fai un \save{FOR} o Perdi d4 FOR (se a cavallo o su di in veicolo, vale per quest'ultimi).

\index{Sosta}
\subparagraph{Sosta}: bisogna sostare per un giorno ogni 6 giorni di viaggio o si deve fare una Prova di Sfinimento per ogni giorno extra di viaggio.

\index{Ingombro}
\index{Marce Forzate}
\index{Gruppi Grandi}
\index{Strade}

\begin{dtable}[Lcc]
	\textbf{Modificatore alla Velocità} & \textbf{Miglia} & \textbf{Griglia} \\

	\textbf{Strade} & +\travelunit{1} & +1 \\

	\mbox{\textbf{Marcia Forzata} per 2 ore extra,} \mbox{richiede una Prova di Sfinimento} & +\travelunit{1} & +1 \\

	\textbf{Gruppi Grandi} & --\travelunit{1} & --1 \\
	
	\mbox{\textbf{Ingombro} oltre le 50~libbre a piedi,} cavalcatura o veicolo sovraccarico & --\travelunit{1} & --1 \\

	\textbf{Condizioni Meteo Rigide} & --\travelunit{1} & --1 \\
	
	\textbf{Condizioni Meteo Estreme} & --\travelunit{2} & --2 \\
	
	\textbf{Attività Concomitanti}
	(esplorare, aggirarsi furtivamente, andare in cerca di cibo, ecc.) & --\travelunit{2} & --2 \\

	\hline

	\multicolumn{3}{l}{\textbf{In Sella:}} \\

	\hspace{0.5em}\labelitemi~Cavalli su di un terreno libero & +\travelunit{1} & +1 \\

	\hspace{0.5em}\labelitemi~Camelli in un deserto & +\travelunit{1} & +1 \\

	\hspace{0.5em}\labelitemi~Elefanti in una giungla & +\travelunit{1} & +1 \\
	
	\hspace{0.5em}\mbox{\labelitemi~Terreno Accidentato o Difficile} \mbox{(eccetto che per asini e muli)} & --\travelunit{1} & --1 \\

	\hline
	
	\multicolumn{3}{l}{\textbf{Veicoli:}} \\

	\hspace{0.5em}\labelitemi~Terreno Accidentato & --\travelunit{1} & --1 \\
	
	\hspace{0.5em}\labelitemi~Terreno Difficile & --\travelunit{2} & --2 \\
\end{dtable}

\vfill

\subparagraph{Passeggeri:} le persone a bordo occupano \sfrac{1}{8}~t dello spazio di carico. I valori per spazio di carico e cavalieri/passeggeri si escludono a vicenda.

\index{Cavalcature}
\index{Mulo}
\index{Asino}
\index{Cavallo}
\index{Camello}
\index{Elefante}
\begin{dtable}[Lccl]
	\textbf{Cavalcatura} & \textbf{Carico} & \textbf{Cavalieri} & \textbf{Costo} \\
	Mulo, Asino	& \sfrac{1}{5}~t (400~lb)	& 1	& 20s \\
	Cavallo, Camello	& \sfrac{1}{4}~t (500~lb)	& 2 & 1f \\
	Elefante		& 2~t (4000~lb)	& 8	& 5f \\
\end{dtable}

\vfill

\index{Veicoli}
\index{Carretto}
\index{Carrozza}
\index{Carrozzone}
\begin{dtable}[Lcccl]
	\textbf{Veicolo} & \textbf{Cavalli} & \textbf{Carico} & \textbf{Passeggeri} & \textbf{Costo} \\
	Carretto	& 1	& \sfrac{1}{2}~t& 4	& 30s \\
	Carrozza& 2 & 1~t			& 8	& 60s \\
	Carrozzone	& 4	& 2~t			& 16& 1f \\
\end{dtable}

\break

\subparagraph{Perdersi} è sempre possbile quando si attraversano terreni estranei o pesantemente oscurati, in un densa nebbia o una pioggia intensa, etc. Se potete avvalervi di una qualche circostanza vantaggiosa nella traversata, tirate un d6; viceversa, tirate un d4:

\begin{dtable}[cL]
	\textbf{d6} & \textbf{Esito} \\
	1	& \textbf{I personaggi si sono persi!} Vagano verso un luogo sconosciuto. \\
	2	& \textbf{I personaggi girano in cerchio.} Per oggi, il viaggio non fa progressi. \\
	3	& \textbf{I personaggi avanzano facendo un percorso tortuoso.} La distanza percorsa va dimezzata. \\
	4--6& \textbf{I personaggi sono sulla buona strada.} \\
\end{dtable}

\subparagraph{Orizzonte:} l'orizzonte dista 3 miglia a livello del mare, 6 miglia a 25~piedi di altezza, (tetto, collina), 12 miglia a 100~piedi (albero della nave, cima di un albero, torre). Il paesaggio può ostruire la visuale. Gli oggetti alti possono essere scorti dietro l'orizzonte.

\dimage{waterborne2}{70pt}

\vspace{-1ex}

\begin{comment}
\subparagraph{Horizon} is 3 miles away on a flat surface (for an Earth-sized planet), 12 miles at 100~ft height (ship's mast, tower), etc.: $dist.~(miles) \approx \sqrt{1.5 \times height~(ft)}$.

\vfill
\dimage{waterborne}{75pt}
\end{comment}

\begin{comment}
\subparagraph{Horizon} is 3 miles away for a 6~ft tall observer (on an Earth-sized planet), 12 miles at 100~ft height (ship's mast, tower), etc.:

\vspace{1ex}
\begin{vwcol}[widths={0.275, 0.725}, rule=0pt]

\vspace{-14pt}\noindent\hspace{0.25em}
$d \approx \sqrt{1.5 \times h}$

\vspace{6pt}\noindent
$$x \approx \frac{(d-o)^2}{1.5}$$

\columnbreak

$d$ --- horizon distance, miles

$h$ --- observer height, feet

$x$ --- object height obscured, feet

$o$ --- object distance, miles

\end{vwcol}

\vfill
\end{comment}

\subsection{Viaggio su Acqua}
\index{Viaggio!su acqua}
\index{Viaggio su Acqua|see {Viaggio, su acqua}}
\index{Imbarcazioni}
\index{Navi}

Si viaggia per \textbf{12 ore/giorno}. Con il doppio dell'equipaggio è possibile fare dei cambi di turno e continuare a viaggiare di notte. Con un equipaggio dimezzato o inferiore, la velocità si dimezza.

\index{Veicoli!su acqua}
\index{Zattera}
\index{Barca a Remi}
\index{Barca a Vela}
\index{Barca a Chiglia}
\index{Veliero}
\index{Nave Lunga}
\index{Galea}
\begin{dtable}[lCCCCL]
\textbf{Veicolo} & \textbf{M} & \textbf{G} & \textbf{E} & \textbf{S} & \textbf{C} \\
Zattera (100~piedi\textsuperscript{2})& \travelunit{2}&2 & 1& \sfrac{1}{4}~t & --- \\
Barca a Remi		& \travelunit{3}	& 3	& 1		& 1~t	& 50s \\
Barca a Vela	& \travelunit{12}	& 12& 1		& 5~t	& 15f \\
Barca a Chiglia	& \travelunit{6}	& 6	& 10	& 20~t	& 25f \\
Nave Lunga	& \travelunit{18}	& 18& 50	& 10~t	& 100f \\
Veliero& \travelunit{18}	& 18& 10	& 100~t	& 150f \\
Galea		& \travelunit{18}	& 18& 100	& 150~t	& 200f \\
\end{dtable}
{\em (M = miglia, G = griglia, E = equipaggio, S = spazio di carico, C = costo)}

Barche a chiglia, navi lunghe e galee hanno sia vele che remi, ma non possono andare a vela contro vento.

\subparagraph{Distanza Coperta:} dipende da meteo e altri fattori. Risalire la corrente riduce la distanza di \travelunit{2} miglia/giorno e navigare con la corrente a favore la incrementa della stessa quantità. Le zattere di fortuna si muovono solo con corrente a favore e con la velocità di quest'ultima.

\index{Tariffe}
\subparagraph{Tariffe:} potrebbero variare da 1p a persona per attraversare un fiume o un lago a 1s a persona per ogni 5 miglia percorse in un viaggio a lunga percorrenza.

\subparagraph{Passeggeri:} occupano 1~t dello spazio di carico o la metà per viaggi a breve percorrenza.

\subparagraph{Razioni:} cibo e acqua per una persona occupano \sfrac{1}{10}~t (200~lb) dello spazio di carico per mese di viaggio.

\index{Macchine d'Assedio}
\subparagraph{Macchine d'Assedio:} potrebbero essere pizzate su barche a chiglia (1), velieri (2) e galee (3).

\break

\subsection{Meteo}
\index{Meteo}
Ricordate che condizioni meteo diverse potrebbero richiedere ritocchi alle tabelle. Per esempio, per climi secchi potreste voler usare la tabella del Cielo usando un d8 o un d12 o un d12~+~8 per quelli piovosi.

Per stabilire quanti giorni durano le attuali condizioni meteo, va scelto un dado appropriato dal d4 al d12, in base a tipo di clima e di meteo.

\vfill
\begin{dtable}[cL|cl]
	\textbf{d20} & \textbf{Cielo} & \textbf{d20} & \textbf{Cielo} \\
	1--4 & sereno	& 13--14 & pioviggine o nebbia \\
	5--8 & nuvoloso	& 15--18 & pioggia o neve \\
	9--12& coperto & 19--20 & tempesta o tormenta \\
\end{dtable}

\vfill
\vfill
\begin{dtable}[cl|cl]
	\textbf{d6} & \textbf{Temperatura} & \textbf{d8} & \textbf{Direzione del Vento} \\
	1	& più fredda del solito&1--3& vento avverso \\
	2--5& normale&4--5& vento laterale \\
	6	& più calda del solito&6--8& vento favorevole \\
\end{dtable}
Quando si segue la direzione del vento dominante, tirate 2d8 e prendete il risultato più alto; quando si va contro di essa, prendete il risultato più basso.

\vfill
\subparagraph{Forza del Vento:} potrebbe influenzare la velocità della navigazione a vela.

\begin{dtable}[cLcc]
	~ & \textbf{Vento} & \multicolumn{2}{c}{\textbf{Moltiplicatore di Navigazione}} \\
	\textbf{d20} & \textbf{Forza} & \textbf{Avverso o Laterale} & \textbf{Favorevole} \\
	1--2	& calmo		& $\times$0 & $\times$0 \\
	3--6	& brezza	& $\times$\sfrac{1}{3}	& $\times$\sfrac{1}{2} \\
	7--14	& medio		& $\times$\sfrac{1}{2}	& $\times$1 \\
	15--18	& forte		& $\times$\sfrac{2}{3}	& $\times$1\sfrac{1}{2} \\
	19--20	& burrasca	& $\times$0	& $\times$2 \\
\end{dtable}

\vfill
Le navi esposte a una burrasca in mare aperto tirano per un Danno da Burrasca ogni 6 ore.

\begin{dtable}[cL]
	\textbf{d8} & \textbf{Danno da Burrasca} \\
	1 	& \textbf{Naufragio.} Nave, carico e \sfrac{1}{2} equipaggio, tutto è andato perduto. \\
	2 	& \textbf{Albero rotto.} Velocità di navigazione a vela nulla. \\
	3 	& \textbf{Metà dei remi rotta.} \sfrac{1}{2} velocità di navigazione a remi. \\
	4 	& \textbf{Vela lacerata.} \sfrac{1}{2} velocità di navigazione a vela. \\
	5--6& \textbf{Fuori bordo.} Smarriti d6 membri dell'equipaggio. \\
	7--8& \textbf{Tutto bene.} \\
\end{dtable}

\vfill

\subparagraph{Meteo Rigido} potrebbero ostacolare la visione, il combattimento a distanza e impedire il Riposo fino al raggiungimento di un riparo.

\subparagraph{Meteo Estremo} (bufera, grandine, ecc.) potrebbero anche infliggere Danno continuato (di solito d4/ora).

\index{Nuotare}
\subparagraph{Nuotare} in acque tempestose richiede un \save{DES}. Fallendo, il soggetto viene è sommerso e deve trattenere il respiro finché non supera un \save{DES} al prossimo turno.

\index{Trattenere il Respiro}
\index{Respirazione|see {Trattenere il Respiro}}
\subparagraph{Trattenere il Respiro} è possibile per FOR/5 turni (arrotondando per eccesso). I turni spesi in compiti gravosi contano per due. Dopodiché si prendono d6~Danno per ciascun turno trascorso senza respirare.

\break

\subsection{Viaggio in Volo}
\index{Viaggio!in volo}
\index{Viaggio in Volo|see {Viaggio, in volo}}

Le creature volanti viaggiano per 8 ore/giorno prima di sostare per la notte. Gli oggetti magici volanti hanno energia per funzionare per la stessa quantità giornaliera di tempo.

\index{Cavalcature}
\begin{dtable}[llCCC]
	\textbf{Cavalcatura} & \textbf{Esempio} & \textbf{M} & \textbf{G} & \textbf{C} \\
	Piccola				& folletto	& \travelunit{8} & 8 &---\\
	Media 				& arpia		& \travelunit{8} & 8 & 1 \\
	Grande				& grifone	& \travelunit{16} & 16 & 2 \\
	Grande e veloce		& pegaso	& \travelunit{24} & 24 & 2 \\
	Enorme				& drago		& \travelunit{16} & 16 & 8 \\
	Dispositivo magico	& scopa		& \travelunit{16} & 16 & 2 \\
	Veicolo magico		& tappeto	& \travelunit{8} & 8 & 8 \\
\end{dtable}
{\em (M = miglia, G = griglia, C = cavalieri)}

La piena velocità è possibile solo con un \sfrac{1}{2}~dei cavalieri o meno; viceversa, la velocità è dimezzata.

\index{Veicoli!volanti}
\subparagraph{Veicoli Volanti:} i veicoli in grado di volare viaggiano per 12 ore/giorno. Il doppio equipaggio permette di continuare a viaggiare di notte.

\index{Mongolfiera}
\index{Dirigibile}
\begin{dtable}[lCCCCL]
\textbf{Veicolo} & \textbf{M} & \textbf{G} & \textbf{E} & \textbf{S} & \textbf{C} \\
Mongolfiera		&	40	& 8	& 1	& 1~t & 25f \\
Dirigibile		&	40	& 8	& 10	& 10~t	& 200f \\
\end{dtable}
{\em (M = miglia, G = griglia, E = equipaggio, S = spazio di carico, C = costo)}

Mongolfiere e dirigibili subiscono l'influenza del vento nella stessa maniera delle navi a vela.

\subparagraph{Nota:} le mongolfiere seguono sempre la direzione del vento. Ogni 3 ore di viaggio si potrebbe cambiare altitudine per cercare di catturare un vento favorevole (la direzione di un nuovo vento va stabilita con un tiro).

\vfill

\subsection{Muoversi in Combattimento ed Esplorare}
\index{Combattimento!movimento}
\index{Movimento}
\index{Esplorazione}
\index{Turni}
Ogni \textbf{turno di combattimento (1 minuto)}, i personaggi si muovono $\times$ 10 piedi del valore della loro velocità su Griglia (di solito, \textbf{30 piedi}; $\pm$10~piedi su terreno sgombro o difficile; $\times$\sfrac{1}{2} quando sono ingombrati; $\times$1\sfrac{1}{2} quando rinunciano a qualunque azione in quel turno).

Ai fini del tracciamento del tempo, le \textbf{attività di esplorazione} richiedono \textbf{10 minuti}: lanciare Incantesimi al di fuori di un combattimento, mettersi in cerca di qualcosa, scassinare serrature, Riposare, ecc.

\vfill

\begin{dbox}
	\index{Unità di Misura}
	\index{Distanza|see {Unità di Misura}}
	\index{Volume|see {Unità di Misura}}
	\index{Peso|see {Unità di Misura}}
	\subsection*{Unità di Misura}
	
	\subparagraph{Distanza}
	\begin{itemize}
		\item \textbf{1 miglio} è 1760 iarde o 5280 piedi (ca. 1,5 km)
		\item \textbf{1 iarda} è 3 piedi o 36 pollici (ca. 1 m)
		\item \textbf{1 piede} è 12 pollici (ca. 30 cm)
	\end{itemize}

	\subparagraph{Volume}
	\begin{itemize}
		\item \textbf{1 gallone} è 4 quarti o 8 pinte (ca. 4 l)
		\item \textbf{1 quarto} è 2 pinte o 32 once (ca. 1 l)
		\item \textbf{1 pinta} è 16 once (ca. 0,5 l)
	\end{itemize}
	
	\subparagraph{Peso}
	\begin{itemize}
		\item \textbf{1 tonnellata} è 2000 libbre (ca. 1000 kg)
		\item \textbf{1 libbra} è 16 once (ca. 0,5 kg)
		\item \mbox{\textbf{1 libbra} è} 100 fiorini d'oro, 1000 scellini d'argento o 1000 pennies di rame
	\end{itemize}
\end{dbox}

\break

\end{document}
