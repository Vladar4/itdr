\documentclass[itdr]{subfiles}

\begin{document}

\chapter{Magia}
\label{ch:magia}
\index{Magia}

\paragraph{Incantesimi}
\index{Runico}
\index{Pergamene}
\index{Incantesimi}
\index{Tomo}
Scritti in Runico e trovati su Tomi e Pergamene, gli Incantesimi richiedono qualche minuto di calma ininterrotta e d’attenzione per poter essere lanciati, oltre a un corredo di gesti dettagliati e cantilene. Pertanto, è di norma impossibile lanciarli in combattimento.

Un Mistico può lanciare qualunque Incantesimo di un Cerchio \textbf{uguale o inferiore} al suo Livello da Mistico.

\index{Incantesimi!in corso}
\subparagraph{Effetti In Corso:} durano finché non si lancia un altro Incantesimo; gli esseri extraplanari convocati restano.

\index{Incantesimi!persistente}
\index{Incantesimi Persistenti|see {Incantesimi, persistente}}
\subparagraph{Incantesimi Persistenti:} l'effetto dura finché si desidera o finché non si lancia di nuovo lo stesso Incantesimo. È possibile tenere contemporaneamente attivi fino a \textbf{2~$\times$~Livello da Mistico} Incantesimi Persistenti.

\index{Incantesimi!trucchetti}
\index{Trucchetti|see {Incantesimi, trucchetti}}
\subparagraph{Trucchetti:} sono trucchi di magia minori e non veri Incantesimi, per questo non interrompono gli effetti in corso di un Incantesimo precedente e il loro lancio richiede l’uso di un \textbf{Focus}.

\vfill
\index{Focus}
\paragraph{Focus}
Ogni Mistico trasporta un Focus, di solito un globo, una bacchetta o un bastone, che consente al costo di un’\textbf{azione} di lanciare istantaneamente, eseguendo i gesti appropriati e le cantilene necessarie, un Incantesimo Preparato o un qualsiasi Trucchetto conosciuto.

Un Focus non funziona quando si indossa un'\textbf{armatura}, \mbox{eccetto} che con i Trucchetti.

\vfill
\index{Incantesimi!dissolvere}
\index{Dissolvere la Magia|see {Incantesimi!dissolvere}}
\paragraph{Dissolvere la Magia}
Il Focus può essere impiegato per dissolvere l’effetto in corso dell’Incantesimo di un altro Mistico a meno che quest’ultimo non superi un \save{VOL}; se il suo Livello da Mistico è più alto del tuo, allora ottiene un Vantaggio. 

Dissolvere un Incantesimo Persistente richiede invece che sia \textbf{tu} a superare un \save{VOL}. In caso di fallimento, l’Incantesimo rimane e tu subisci una Perdita di~VOL in base al Cerchio dell’Incantesimo: da d2 (Trucchetto) a d12 (5° Cerchio); se il tuo Livello da Mistico è più alto del Cerchio dell’Incantesimo, ottieni un Vantaggio.

\vfill
\begin{dbox}
	Le Pergamene di nuovi Incantesimi trovate dai Mistici vengono di solito allegate al Tomo per facilità d’uso.

	Vedi \textbf{\safenameref{sec:fabbricare_equipaggiamento_magico}{Fabbricare Equipaggiamento Magico}} nell'\mbox{\textbf{\customref{ch:appendice_a}{Appendice A}}} per ulteriori informazioni sulla creazione di Focus e Pergamene.
\end{dbox}

\vfill
\index{Attivazione delle Pergamene}
\begin{dbox}
	\paragraph{Attivazione delle Pergamene (opzionale)}
	Come sua azione, qualsiasi personaggio può attivare una Pergamena. L’Incantesimo è lanciato come Incantesimo Distintivo e distrugge la Pergamena. Bisogna superare un \save{VOL} o si subisce un \textbf{Contrattempo Magico} (vedi \textbf{\safenameref{sec:contrattempi_magici}{Contrattempi Magici}} nell'\textbf{\customref{ch:appendice_a}{Appendice A}}).
\end{dbox}

\vfill

\index{Incantesimi!preparato}
\index{Incantesimi Preparati|see {Incantesimi, preparato}}
\index{Arcanocombustione}
\paragraph{Incantesimi Preparati e Arcanocombustione}
Durante il Riposo puoi leggere il tuo Tomo e preparare tanti Incantesimi quant'è il tuo Livello d’Esperienza, così da poterli lanciare con un’azione usando il Focus.

Lanciare un Incantesimo Preparato come azione infligge, su \textbf{chi lo lancia}, \mbox{Danno} da \mbox{\textbf{Arcanocombustione}}, \textbf{che ignora l’Armatura} e che è pari a \textbf{2pf per Cerchio dell’Incantesimo}. A 0pf, l’Arcanocombustione bersaglia la VOL \mbox{invece} che la FOR: bisogna quindi superare un \save{VOL} per evitare Arcanocombustione Critica oppure si prende la condizione Stordito per il prossimo turno.

\begin{dbox}
	Facoltativamente, ciò provocherà anche un \textbf{Contrattempo Magico} (vedi \textbf{\safenameref{sec:contrattempi_magici}{Contrattempi Magici}} nell'\textbf{\customref{ch:appendice_a}{Appendice A}}).
\end{dbox}

\vfill
\index{Incantesimi!distintivo}
\index{Incantesimi Distintivi|see {Incantesimi, distintivo}}
\paragraph{Incantesimi Distintivi}
Ogni volta che prendi il Tratto Mistico, scegli quale dei tuoi Incantesimi vuoi rendere Distintivo. Un Incantesimo Distintivo si può lanciare senza preparazione e alla metà del suo costo (1pf per Cerchio dell’Incantesimo).

\vfill
\dimage{magic4}{370pt}

\break

\begin{dbox}
	\index{Incantesimi!distanza}
	\index{Incantesimi!area}
	\index{Aree degli Incantesimi|see {Incantesimi, area}}
	\subsection*{Distanze/Aree}
	\begin{itemize}
		\item \textbf{Breve o Ravvicinata/Piccola} --- un paio di passi
		\item \textbf{Intermedia} --- circa 30~piedi (il movimento in un turno)
		\item \textbf{Distante/Grande o Nei Paraggi} --- a circa 60~piedi
	\end{itemize}
	\vspace{1ex}
	\hrule
	\vspace{1ex}
	Quando non specificato, l’Incantesimo influenza un singolo bersaglio nei paraggi e che sei in grado di vedere.
\end{dbox}

\vfill

\begin{dbox}
\subsection*{Incantesimi Random}
\phantomsection
\label{incantesimi_random}
\index{Incantesimi!random}
\index{Incantesimi Random|see {Incantesimi, random}}

\begin{comment}
~\\
\header{Random 30 Spells}
\begin{dtable}[L]
d3~$\times$10~+~d10 (treat 10 as 0)
\end{dtable}
\end{comment}

~\\
\header{36 Incantesimi Random}
\begin{dtable}[cC|cC|cC]
	\textbf{d6,d6} & \textbf{Inc.} & \textbf{d6,d6} & \textbf{Inc.} & \textbf{d6,d6} & \textbf{Inc.} \\

	1,1 & 1 & 3,1 & 13 & 5,1 & 25 \\
	1,2 & 2 & 3,2 & 14 & 5,2 & 26 \\
	1,3 & 3 & 3,3 & 15 & 5,3 & 27 \\
	1,4 & 4 & 3,4 & 16 & 5,4 & 28 \\
	1,5 & 5 & 3,5 & 17 & 5,5 & 29 \\
	1,6 & 6 & 3,6 & 18 & 5,6 & 30 \\
	\hline
	2,1 & 7 & 4,1 & 19 & 6,1 & 31 \\
	2,2 & 8 & 4,2 & 20 & 6,2 & 32 \\
	2,3 & 9 & 4,3 & 21 & 6,3 & 33 \\
	2,4 & 10 & 4,4 & 22 & 6,4 & 34 \\
	2,5 & 11 & 4,5 & 23 & 6,5 & 35 \\
	2,6 & 12 & 4,6 & 24 & 6,6 & 36 \\
\end{dtable}
{\em (Inc. = Incantesimo)}

~\\

\header{40 Incantesimi Random}
\begin{dtable}[L]
d4~$\times$10~+~d10 (sul d10, considera 10 uguale a 0)
\end{dtable}

~\\
\header{48 Incantesimi Random}
\begin{dtable}[cC|cC|cC]
	\textbf{d6,d8} & \textbf{Inc.} & \textbf{d6,d8} & \textbf{Inc.} & \textbf{d6,d8} & \textbf{Inc.} \\

	1,1 & 1 & 3,1 & 17 & 5,1 & 33 \\
	1,2 & 2 & 3,2 & 18 & 5,2 & 34 \\
	1,3 & 3 & 3,3 & 19 & 5,3 & 35 \\
	1,4 & 4 & 3,4 & 20 & 5,4 & 36 \\
	1,5 & 5 & 3,5 & 21 & 5,5 & 37 \\
	1,6 & 6 & 3,6 & 22 & 5,6 & 38 \\
	1,7 & 7 & 3,7 & 23 & 5,7 & 39 \\
	1,8 & 8 & 3,8 & 24 & 5,8 & 40 \\
	\hline
	2,1 & 9 & 4,1 & 25 & 6,1 & 41 \\
	2,2 & 10 & 4,2 & 26 & 6,2 & 42 \\
	2,3 & 11 & 4,3 & 27 & 6,3 & 43 \\
	2,4 & 12 & 4,4 & 28 & 6,4 & 44 \\
	2,5 & 13 & 4,5 & 29 & 6,5 & 45 \\
	2,6 & 14 & 4,6 & 30 & 6,6 & 46 \\
	2,7 & 15 & 4,7 & 31 & 6,7 & 47 \\
	2,8 & 16 & 4,8 & 32 & 6,8 & 48 \\
\end{dtable}
{\em (Inc. = Incantesimo)}
\end{dbox}

\vfill
\break

L’Arbitro potrebbe fornire ai Mistici un elenco degli Incantesimi per i loro Tomi o usare gli esempi in basso, tenuto conto che essi sono ben lungi dal rappresentare tutti gli Incantesimi che esistono al mondo, la vasta maggioranza dei quali resta sconosciuta a qualunque persona.

\vfill

\section{Trucchetti}
\index{Incantesimi!trucchetti}
\def \spellcircle {T}
\begin{enumerate}
	\item \spell{Accendere/Spegnere} un oggetto che hai in mano illumina come una torcia oppure viene spenta una fonte di luce nei paraggi non più grande di una torcia.
	\item \spell{Colpo Guidato} il bersaglio ottiene un dado bonus per il Danno al suo prossimo attacco.
	\item \spell{Frastornare} una creatura umanoide deve superare un \save{VOL} o viene Stordita per il prossimo turno.
	\item \spell{Iettatura} il prossimo attacco di una creatura umanoide è Compromesso.
	\item \spell{Individuazione del Magico} individua gli effetti degli Incantesimi e gli oggetti magici nei paraggi (l’individuazione è bloccata da pareti, porte, ecc.).
	\item \spell{Individuazione del Veleno} individui con il tuo tocco la presenza di veleno in una creatura o in un piccolo oggetto.
	\item \spell{Mano Magica} telecinesi fino a 5 libbre.
	\item \spell{Marchio Arcano} incidi una runa personale (visibile o invisibile). Persistente.
	\item \spell{Occultare} tocchi un oggetto che potresti tenere nel palmo di una mano e lo rendi Invisibile. Persistente.
	\item \spell{Ostacolare la Non Morte} infligge d4 Danno a un essere \mbox{non morto}, ignorando Armatura e resistenze.
	\item \spell{Prestidigitazione} esegue un semplice trucco di magia, crea o cela un effetto sensoriale minore.
	\item \spell{Provocazione} una creatura deve superare un \save{VOL} o è spinta ad attaccarti.
	\item \spell{Raggio di Gelo} un raggio fa d4 Danno da Freddo.
	\item \spell{Resistenza} al tuo tocco, una creatura ignora gli effetti di norma fastidiosi, come caldo soffocante, malattie cutanee pruriginose o una tempesta di sabbia. Persistente.
	\item \spell{Riparare} esegui con un tocco riparazioni minori su di un oggetto.
	\item \spell{Rumore Fantasma} diffonde voci o suoni di fantasia oppure sussurra un messaggio a un bersaglio in vista.
	\item \spell{Scintilla} il tuo tocco fa d4 Danno da Elettricità che ignora l’Armatura.
	\item \spell{Segnale Luminoso} invia un segnale luminoso che può essere visto da una certa distanza. Se scagliato contro un bersaglio fa d4 Danno da Fuoco.
	\item \spell{Spruzzo Acido} un globo fa d4 Danno da Acido. Corrode leggermente il legno.
	\item \spell{Tacitare} al tuo tocco, una creatura umanoide viene resa muta per il prossimo turno.
\end{enumerate}

\index[Incantesimi]{Spegnere|see {Accendere/Spegnere}}

\vfill
\break

\section{1° Cerchio}
\index{Incantesimi!1° Cerchio}
\def \spellcircle {1}
\begin{enumerate}
	\setcounter{enumi}{9}
	\item \spell{Allarme} i soggetti intrusi attivano un allarme udibile solo da te. Persistente.
	\item \spell{Animare Corda} muovi una corda al comando.
	\item \spell{Auto-Mascheramento} ti cambi l’aspetto del viso.
	\item \spell{Blocca Porta} mantiene chiusa una porta.
	\item \spell{Caduta Piume} gli oggetti o le creatura in una piccola sfera cadono lentamente.
	\item \spell{Camuffamento} chiunque si trovi adiacente a te è difficile da notare o da rintracciare.
	\item \spell{Cancella} scritti normali o magici svaniscono al tuo tocco.
	\item \spell{Charme} \save{VOL} o un essere umanoide diventa amichevole fino al suo prossimo Riposo.
	\item \spell{Colla} un singolo oggetto viene incollato a un altro. Se lanciato su di una creatura, deve superare un \save{FOR} o non può più muoversi.
	\item \spell{Comprensione dei Linguaggi} comprendi tutti i linguaggi parlati e scritti.
	\item \spell{Convoca Creatura} chiama una creatura extraplanare non intelligente non più grande di un piccolo cane. Non ha vincoli di lealtà verso di te.
	\item \spell{Coraggio} finché non Riposa, una creatura consenziente è immune alla paura, ma incapace di ritirarsi da una battaglia.
	\item \spell{Dardo Incantato} svolta gli angoli, d4 Danno che ignora l’Armatura.
	\item \spell{Destriero} convoca un cavallo da corsa. Sparisce dopo che ha preso Danno.
	\item \spell{Disco Fluttuante} crea un disco orizzontale del diametro di 3~piedi, che può reggere 100~libbre, fluttua a 3~piedi dal suolo e si muove piano al tuo comando.
	\item \spell{Folata di Vento} soffia via o butta giù cose in un cono medio. \save{FOR} per resistere.
	\item \spell{Foschia Coprente} della foschia copre una piccola area attorno a te. Gli attacchi a distanza sono Compromessi.
	\item \spell{Grasso} rende scivolosa una cosa o una piccola area. \save{DES} per evitare scivolamenti.
	\item \spell{Identifica} svela le proprietà base di un oggetto magico in mano, come modo di attivazione ed effetto generale. Info precise su come funziona, proprietà nascoste, maledizioni, ecc., non sono rilevate.
	\item \spell{Immagine Silenziosa} crea un’\mbox{illusione} minore immobile di tua ideazione.
	\item \spell{Incuti Paura} \save{VOL} o la creatura fugge per la durata.
	\item \spell{Individuazione dei Caduti} rivela corpi morti ed esseri non morti nei paraggi.
	\item \spell{Individuazione delle Porte Segrete} rivela le porte nascoste nei paraggi.
	\item \spell{Ingrandisci/Rimpicciolisci} al tuo tocco, una creatura umanoide raddoppia o dimezza la sua taglia rispettivamente Incrementando o Riducendo il dado del Danno della sua arma. Il bersaglio potrebbe scegliere di evitare l’effetto con un \save{FOR}.

\vfill
\break

	\item \spell{Ipnotismo} affascina d6 creature che falliscono un \save{VOL}. In combattimento, i loro attacchi al prossimo turno sono Compromessi.
	\item \spell{Mani Brucianti} d6 Danno da Fuoco in un piccolo cono.
	\item \spell{Pirotecnica} diffondi o estingui il fuoco, tramutandolo in luce accecante o fumo soffocante.
	\item \spell{Protezione} ignori il prossimo caso di effetti nocivi provenienti da una specifica fonte.
	\item \spell{Risata Orrida} \save{FOR} o una creatura umanoide ride e Compromette i suoi attacchi finché non supera il TS alla fine del suo turno.
	\item \spell{Ritirata Rapida} corri al doppio della velocità.
	\item \spell{Salto} una creatura può saltare il doppio della distanza in lungo e in altezza.
	\item \spell{Sciame} convoca uno sciame di pipistrelli, ratti o ragni. È innocuo, ma distrae.
	\item \spell{Scudo} un disco invisibile ti dà +1 Armatura e blocca i Dardi Incantati.
	\item \spell{Servitù Invisibile} una forza invisibile (FOR~5, 1pf, non può attaccare) obbedisce al tuo comando.
	\item \spell{Sonno} ha effetto su d6 creature viventi. Quelle rilassare si addormentano, le altre si sentono letargiche, il che Riduce i loro dadi del \mbox{Danno}.
	\item \spell{Sopportare Elementi} chiunque si trovi adiacente a te può rimanere a suo agio in ambienti caldi o freddi.
	\item \spell{Spray Colorato} \save{DES} o il bersaglio è Accecato al prossimo turno.
	\item \spell{Stretta Folgorante} il tuo tocco fa d6 \mbox{Danno} da Elettricità che ignora l’Armatura.
	\item \spell{Tocco Gelido} \save{FOR} o una creatura vivente Perde~d4~FOR.
	\item \spell{Vero Colpo} il bersaglio del tuo prossimo attacco deve superare un \save{DES} o l’attacco bypassa i PF e va direttamente sul Punteggio di FOR.
\end{enumerate}

\index[spells]{Rimpicciolisci|see {Ingrandisci/Rimpicciolisci}}

\vfill
\break

\section{2° Cerchio}
\index{Incantesimi!2° Cerchio}
\def \spellcircle {2}
\begin{enumerate}
	\item \spell{Accecare} \save{FOR} o si prende la condizione Accecato finché non ci si Riposa.
	\item \spell{Arma Magica} tocchi un'arma per renderla magica (Incrementi il dado del Danno (fino a d10), ignora tutte le resistenze soprannaturali) per la durata.
	\item \spell{Assordare} assorda chiunque in'area intermedia.
	\item \spell{Altera Sembianze} assumi le sembianze di una creatura simile a te.
	\item \spell{Bocca Magica} tocchi un oggetto e lo fai parlare una volta o ogni volta che è innescato. Persistente.
	\item \spell{Comandare Essere Non Morto} una creatura non morta ti obbedisce se non supera un \save{VOL}.
	\item \spell{Convoca Animale} chiami un animale extraplanare intelligente. Non ha vincoli di lealtà verso di te.
	\item \spell{Eroismo} una volta prima di ciascun Riposo, una creatura può tirare di nuovo un dado del Danno o un 20 ottenuto con un TS. Persistente.
	\item \spell{Falsa Vita} riottieni tutta la FOR Persa, ma svanisce dopo un minuto o se lanci un altro Incantesimo.
	\item \spell{Fiamma Continua} l'oggetto toccato emette luce come una torca perenne che non fa calore. Persistente.
	\item \spell{Forza del Toro} dà un d8 per il Danno senz'armi in mischia e Vantaggio sui \saves{FOR}.
	\item \spell{Frantumare} una vibrazione sonora fa d6 Danno da Scoppio che ignora l'Armatura a chi ti è adiacente. Oggetti o esseri di cristallo subiscono invece d12 Danno da Scoppio che ignora l'Armatura.
	\item \spell{Freccia Acida} d6 Danno da Acido ora e, se non è lavato via, d4~Punti di FOR Persi (soggetti all'Armatura) alla fine del prossimo turno.
	\item \spell{Grazia del Gatto} dà Vantaggio ai \saves{DES}, i dadi del Danno delle armi a distanza sono Incrementati.
	\item \spell{Illusione Minore} evoca un'immagina con suono.
	\item \spell{Immagine Speculare} crea d4 esche che sono tuoi duplicati. Un duplicato sparisce se viene colpito.
	\item \spell{Individuazione dei Pensieri} \save{VOL} o si \mbox{``odono''} i pensieri superficiali del bersaglio.
	\item \spell{Invisibilità} un bersaglio è Invisibile finché non attacca.
	\item \spell{Ira} gli attacchi della creatura sono Potenziati, ma lo sono anche gli attacchi contro di essa.
	\item \spell{Levitazione} il bersaglio si sposta su e già come vuole, per poi fluttuare giù in sicurezza. \mbox{\save{VOL} per far levitare chi è più pesante di te}.
	\item \spell{Localizza Oggetto} indirizza verso l'oggetto.
	\item \spell{Maleficio} infligge uno Svantaggio al prossimo TS.
	\item \spell{Mano Spettrale} crea un mano incorporea luccicante per recapitare un tuo Incantesimo da contatto come azione in uno dei tuoi prossimi turni.
	\item \spell{Mosse da Ragno} puoi spostarti su soffitti e pareti.
	\item \spell{Nube di Nebbia} della nebbia ostacola la visione in un'area grande. Gli attacchi a distanza attraverso di essa sono Compromessi.
	\item \spell{Oscurità} crea un'area intermedia di ombra soprannaturale.

\vfill
\break

	\item \spell{Parlare a un Cadavere} un cadavere risponde a tre domande prima di ridursi in polvere. Le risposte devono essere sincere, potrebbero essere criptiche e si baseranno su quanto il bersaglio sapeva in vita.
	\item \spell{Polvere Luccicante} \save{DES} o Potenzia gli attacchi contro la creatura bersagliata. Rivela un bersaglio Invisibile.
	\item \spell{Protezione dalle Frecce} al tuo tocco, la creatura diventa immune agli attacchi a distanza non magici.
	\item \spell{Raggio di Indebolimento} \save{DES} o tutti gli attacchi sono Compromessi finché non si Riposa.
	\item \spell{Raggio Rovente} fa d8 Danno da Fuoco.
	\item \spell{Ragnatela} riempie un'area intermedia con appiccicose tele di ragno. \save{FOR} o non ci si può muovere in questo turno.
	\item \spell{Resistenza dell'Orso} una creatura ottiene Armatura~2.
	\item \spell{Resistere a un Elemento} un tipo specifico di Danno elementale preso da una creatura è Compromesso.
	\item \spell{Riscaldare Metalli} riscalda un oggetto di metallo arroventandolo. Ogni turno, il contatto fa d6 Danno da Fuoco.
	\item \spell{Saggezza del Gufo} amplifica i sensi di percezione e dà Vantaggio ai \saves{VOL}.
	\item \spell{Scassinare} un urto rumoroso apre porte e serrature.
	\item \spell{Scurovisione} vedi nel buio naturale nei paraggi.
	\item \spell{Serratura Arcana} serri magicamente una porta o uno scrigno che tocchi. Persistente.
	\item \spell{Sfera Fiammeggiante} crea una sfera rotolante di fuoco, d8 Danno da Fuoco con un \save{DES} fallito. Ogni turno puoi scegliere in che direzione va. Dopo aver inflitto Danno, si ferma per quel turno.
	\item \spell{Sfocatura} i contorni della tua figura non sono più distinguibili. Gli attacchi contro di te sono Compromessi.
	\item \spell{Silenzio} non si può produrre alcun suono in un'area intermedia e ciò preclude il lancio di Incantesimi.
	\item \spell{Tocco del Ghoul} \save{FOR} o il bersaglio è Stordito finché non supera un \save{FOR} alla fine del suo turno; nel mentre emana un fetore che disgusta chi si trova nelle sue vicinanze.
	\item \spell{Tocco di Idiozia} \save{FOR} o si perde d4 VOL.
	\item \spell{Trappola Fantasma} fa sembrare che un oggetto sia protetto da una trappola. Persistente.
	\item \spell{Trucco della Corda} una corda conduce verso uno spazio extradimensionale che può ospitare fino a sei creature.
	\item \spell{Vento Sussurrante} invia un breve messaggio a un gruppo o a un singolo bersaglio conosciuti entro un miglio.
	\item \spell{Vista Arcana} le auree magiche in una sfera intermedia diventano a te visibili, anche attraverso pareti e altri ostacoli, rivelando la maggior parte delle informazioni sulla loro natura.
\end{enumerate}

\vfill
\break

\section{3° Cerchio}
\index{Incantesimi!3° Cerchio}
\def \spellcircle {3}
\begin{enumerate}
	\item \spell{Assorbi Elemento} al tuo tocco, la creatura diventa immune a uno specifico tipo di Danno elementale.
	\item \spell{Blocca Persona} Stordisce una creatura umanoide a meno che non superi un \save{FOR} alla fine del suo turno.
	\item \spell{Capanna Minuscola} crea un rifugio per dieci creature.
	\item \spell{Cerchio Magico} evita che un certo tipo di esseri innaturali (extraplanari, non morti, ecc.) entrino o escano a meno che non superino un \save{VOL}. Adatto a una sola creatura. Persistente.
	\item \spell{Chiaroudienza/Chiaroveggenza} ascolti o vedi dalla distanza o attraverso un muro come se tu fossi lì.
	\item \spell{Dislocazione} ottieni Vantaggio ai TS dovuti a Danno Critico. Persistente.
	\item \spell{Filo Tagliente} il prossimo attacco con questa arma tagliente ignora i PF e va direttamente sul Punteggio di FOR.
	\item \spell{Forma Gassosa} una creatura consenziente diventa \mbox{incorporea} e può volare lentamente. Il bersaglio o chi ha lanciato l'Incantesimo possono porre fine all'effetto quando vogliono. Persistente.
	\item \spell{Frecce Infuocate} le munizioni del bersaglio alleato fanno d6 bonus di Danno da Fuoco.
	\item \spell{Fulmine Magico} d8 Danno da Elettricità che ignora l'Armatura a chiunque nella stessa fila.
	\item \spell{Illusione Maggiore} evoca un immagine con suono, odore ed effetti termici. Si potrebbe anche usare per camuffare l'aspetto di una creatura.
	\item \spell{Immobilizza Non Morte} immbolizza tutti gli esseri non morti nei paraggi che falliscono il loro \save{VOL}.
	\item \spell{Intermittenza} ogni turno hai una possibilità del 50\% di sparire e riapparire nel tuo prossimo turno, evitando così il prossimo attacco contro di te.
	\item \spell{Lingue} puoi parlare qualunque linguaggio.
	\item \spell{Luce Diurna} un'area grande di luce intensa abbastanza luminosa da sopraffare anche il buio magico.
	\item \spell{Muro di Vento} una linea di vento forte fa deviare le frecce, le creatura più piccole e i gas.
	\item \spell{Nube Maleodorante} vapori nauseanti riempiono un'area intermedia. Chiunque si trovi nella nube deve superare un \save{FOR} o vomita, ottenendo Svantaggio al prossimo TS. Il TS va ripetuto all'inizio di ogni turno in cui si sta nell'area.
	\item \spell{Offuscare} nasconde un bersaglio dalla \mbox{divinazione} e dallo scrutamento o confonde tali sforzi. \mbox{Persistente.}
	\item \spell{Palla di Fuoco} fa d10 Danno da Fuoco all'interno di una sfera intermedia.
	\item \spell{Portale di Convocazione} chiama un essere extraplanare che desidera entrare nel tuo piano. Non puoi scegliere quale essere sarà e non ha vincoli di lealtà verso di te.

\vfill
\break

	\item \spell{Respirare Sott'Acqua} le creature scelte da te possono respirare sott'acqua.
	\item \spell{Restringere Oggetto} tocchi un oggetto non magico e lo restringi a un sedicesimo della sua taglia e del suo peso.
	\item \spell{Rune Esplosive} incidi delle Rune che fanno d10 Danno da Scoppio se lette o toccate, per poi sparire. Persistente.
	\item \spell{Scritto Illusorio} tocchi una pagina per cambiare o nascondere il suo reale contenuto, così che solo un soggetto designato possa decifrarla. Persistente.
	\item \spell{Sfera di Invisibilità} il bersaglio e chiunque si trovi in una sfera piccola vicina sono Invisibili finché non attaccano o non si allontanano troppo dal bersaglio.
	\item \spell{Sigillo della Serpe} incidi un piccolo simbolo testuale che Stordisce chi lo legge a meno che non superi un \save{VOL} alla fine del suo turno. Persistente.
	\item \spell{Sogno} invii un messaggio a chiunque stia dormendo.
	\item \spell{Sonno Profondo} fai addormentare d6 creature che falliscono il proprio \save{VOL}; dura finché l'Incantesimo non è spezzato o la creatura non subisce Danno.
	\item \spell{Suggestione} \save{VOL} o il bersaglio che ti comprende è costretto a seguire il piano d'azione pronunciato. I TS contro azioni che sono nocive per il bersaglio sono tirati con Vantaggio.
	\item \spell{Tempesta di Nevischio} in un'area grande, le fiamme vengono spente e gli attacchi a distanza sono Compromessi. \save{DES} per evitare scivolamenti.
	\item \spell{Tentacoli Neri} alcuni tentacoli si avvinghiano a chi fallisce un \save{FOR o DES} all'interno di un'area media, Compromettendo i suoi attacchi finché non supera un \save{FOR} alla fine di uno dei suoi turni.
	\item \spell{Tocco Vampirico} con un \save{FOR} fallito, una creatura vivente perde d6~FOR e tu recuperi tutti i PF.
	\item \spell{Urlo} i soggetti in un cono intermedio sono assordati per un turno e prendono d8 Danno.
	\item \spell{Vedere Invisibilità} rivela le creature e gli oggetti Invisibli nei paraggi.
	\item \spell{Velocità/Lentezza} una creatura si muove rispettivamente al doppio o alla metà della sua velocità di movimento e ottiene +1 o --1~Armatura e Vantaggio/Svantaggio ai \saves{DES}.
	\item \spell{Volare} una creatura vola.
\end{enumerate}

\index[spells]{Chiaroveggenza|see {Chiaroudienza/Chiaroveggenza}}
\index[spells]{Lentezza|see {Velocità/Lentezza}}

\vfill
\break

\section{4° Cerchio}
\index{Incantesimi!4° Cerchio}
\def \spellcircle {4}
\begin{enumerate}
	\item \spell{Allucinazione Letale} una spaventosa e invincibile illusione visibile al solo bersaglio. Attacca facendo d10 Danno, poi sparisce. Con Danno Critico, il bersaglio deve superare un \save{VOL} o muore di paura.
	\item \spell{Ancora Dimensionale} vieta il movimento extradimensionale nei paraggi.
	\item \spell{Animare Corpi} crea da cadaveri fino a d4 zombi e scheletri non morti. Hai il \mbox{controllo} su di essi mentre dura l'Incantesimo.
	\item \spell{Catena di Fulmini} colpisce d10 bersagli infliggendo a ciascuno d10 Danni da Elettricità che ignora l'Armatura.
	\item \spell{Charme Superiore} \save{VOL} o una creatura ti tratta come se tra voi ci fosse un'alleanza.
	\item \spell{Confusione} le creature nell'area intermedia che falliscono un \save{VOL} si comportano in modo strano. In combattimento, tira un d4 a ognuno dei loro turni: 1 = attaccano i soggetti a loro alleati, \mbox{2--3 = non} fanno nulla, 4 = attaccano i soggetti a loro nemici.
	\item \spell{Contagio} infetta una creatura con un'\mbox{orribile} malattia, che toglie d6 punti a un singolo Punteggio di Abilità immediatamente e poi a ogni giorno successivo finché non viene Guarita.
	\item \spell{Convoca Essere} chiama un qualsiasi essere extraplanare scelto. Non ha vincoli di lealtà verso di te.
	\item \spell{Creare Acqua} una fonte inizia a zampillare acqua dal pavimento o da una parete.
	\item \spell{Creazione Minore} crea un piccolo oggetto di tessuto o di legno. Persistente.
	\item \spell{Disperazione Schiacciante} tutti i soggetti in un'area grande devono superare un \save{VOL} o i loro attacchi sono Compromessi.
	\item \spell{Globo di Invulnerabilità} ferma gli Incantesimi fino al 3° Cerchio dentro una piccola sfera.
	\item \spell{Incubo} \save{VOL} ogni notte o il bersaglio si sveglia con la metà dei PF e non riuscirà a ripristinarli finché non dormirà per una notte intera senza fare brutti sogni. Persistente.
	\item \spell{Individuzione dello Scrutamento} ti avverte della presenza di intercettazioni magiche.
	\item \spell{Indurre il Panico} le creature all'interno di un cono grande devono superare un \save{VOL} o fuggono per la durata.
	\item \spell{Infliggi Maledizione} il bersaglio ottiene Svantaggio a tutti i TS finché non viene Guarito.
	\item \spell{Locazza Creatura} indica la direzione verso una creatura familiare.
	\item \spell{Messaggio} consegna istantaneamente e ovunque un breve messaggio. Un soggetto destinatario può inviare una breve \mbox{risposta}.
	\item \spell{Muro di Fuoco} passare attraverso questo grande muro infligge d10 Danno da Fuoco.

\vfill
\break

	\item \spell{Muro di Ghiaccio} crea un grande muro di ghiaccio (12pf, Armatura~3) o una semisfera. Può intrappolare le creature al suo interno, a meno che non superino un \save{DES}.
	\item \spell{Nebbia Solida} impedisce la visione e rallenta i movimenti in un'area grande.
	\item \spell{Occhio Arcano} crea un occhio Invisibile fluttuante che controlli e attraverso cui puoi vedere e anche lanciare i tuoi Trucchetti.
	\item \spell{Pelle di Pietra} una creatura ottiene Armatura~3, ma non può più correre o nuotare.
	\item \spell{Plasma Pietra} scolpisce un piccolo cubo di pietra dandogli una qualsiasi forma.
	\item \spell{Polimorfismo} dà a una creatura consenziente una nuova forma permanente. Il bersaglio mantiene i suoi Punteggi di Abilità e PF mentre ottiene le capacità e le limitazioni della nuova forma, con l'eccezione di poteri soprannaturali, resistenze ecc., e non può essere trasformato di nuovo se non dopo un giorno. Superare un \save{VOL} consentirà di ottenere l'esatto aspetto desiderato, altrimenti varierà in modo casuale.
	\item \spell{Porta Dimensionale} ti teletrasporta entro una distanza intermedia.
	\item \spell{Rimuovi Maledizioni} al tuo tocco, una creatura si libera da qualunque Svantaggio o Compromissione di natura magica.
	\item \spell{Riparo Sicuro} crea un solido cottage.
	\item \spell{Scrutare} spia il bersaglio da lontano.
	\item \spell{Scudo di Fuoco} le creature che ti attaccano in mischia prendono d6 Danno da Fuoco; tu, sei immune al Danno da Fuoco e a quello da Freddo.
	\item \spell{Sfera Resiliente} un globo di forza protegge una creatura, ma la intrappola. Per evitarlo, bisogna superare un \save{DES}.
	\item \spell{Simbolo di Dolore} incidi una piccola Runa che causa dolore quando letta. Il soggetto che la legge perde immediatamente d4~FOR e deve superare un \save{VOL} o viene Stordito e urla finché non supera un \save{VOL} alla fine del suo turno. Persistente.
	\item \spell{Terreno Illusorio} cambia l'aspetto esteriore di un tipo di terreno, parete, pavimento, soffitto, ecc.
	\item \spell{Trama Arcobaleno} delle luci affascinanto le creature che possono vederti. In combattimento, sono Stordite finché non superano un \save{VOL} alla fine del proprio turno. Superato il TS, diventano immuni all'effetto finché l'Incantesimo non viene lanciato di nuovo.
	\item \spell{Trappola di Fuoco} il tuo tocco piazza una trappola su di un aggeggio o una porta. L'apertura infligge d12 Danno da Fuoco. Persistente.
	\item \spell{Vera Invisibilità} una creatura può attaccare e restare Invisibile.
\end{enumerate}

\vfill
\break

\section{5° Cerchio}
\index{Incantesimi!5° Cerchio}
\def \spellcircle {5}
\begin{enumerate}
	\item \spell{Altera Fato} il risultato del prossimo tiro del bersaglio è alterato da un $\pm$d12, ma non può uscire dalla gamma di risultati del tiro originale.
	\item \spell{Arma Perfetta} convoca un'arma da mischia (d10/d12) o a distanza (d10) che ignora tutte le resistenze soprannaturali. Svanisce dopo aver tirato il massimo Danno possibile. Persistente.
	\item \spell{Ceppi Planari} intrappola, finché non esegue un compito, una creatura extraplanare che fallisce un \save{VOL}.
	\item \spell{Compagnia dell'Eremita} convoca un tuo duplicato. È incapace di fare magie, non può nuocere o disubbidirti ed è sempre dell'umore giusto. Qualunque Danno inferto a te, lo subisce anche il tuo duplicato e viceversa. \mbox{Persistente}.
	\item \spell{Cono di Freddo} d12 Danno di Freddo a chiunque si trovi all'interno di un cono grande.
	\item \spell{Contattare Altro Piano} ti fa porre una domanda a un'entità extraplanare. \save{VOL} o perdi d6~VOL.
	\item \spell{Controllo dell'Acqua} innalzi, abbassi o separi l'acqua.
	\item \spell{Creazione Maggiore} crei un oggetto di pietra e metallo. Persistente.
	\item \spell{Demenza Precoce} al tuo tocco, \save{VOL} o si va a VOL~0.
	\item \spell{Disintegrazione} d12 Danno che ignora l'Armatura. Fallendo un TS per Danno Critico, la creatura si polverizza. I bersagli non più grandi di un elefante sono distrutti del tutto a 0pf.
	\item \spell{Dominare Persona} \save{VOL} o una creatura umanoide è controllata per via telepatica. Il TS va ripetuto ogni volta che si nuoce al bersaglio.
	\item \spell{Esilio} \save{VOL} o una creatura ritorna al suo piano d'esistenza natio. Se nativa del piano corrente, sparisce per un minuto e poi ritorna in sicurezza.
	\item \spell{Fedele Segugio del Mistico} un cane fantasma può fare la guardia o attaccare e resterà sempre nei paraggi di chi ha lanciato l'Incantesimo. VOL~15, 3d6pf, d8~Morso. Persistente.
	\item \spell{Grazia Salvifica del Mistico} se il bersaglio prende Danno, puoi decidere di prenderlo tu al suo posto, terminando l'Incantesimo. A 0pf funziona come un Danno da lancio di Incantesimi che bersaglia la tua VOL. Persistente.
	\item \spell{Incenerire} dà fuoco a un bersaglio. Fa d12 Danno da Fuoco ora e poi, finché non si supera un \save{DES} o non si trova un sistema per spegnere le fiamme, alla fine di ogni prossimo turno.
	\item \spell{Legame Telepatico} crea un collegamento telepatico tra soggetti alleati. Tutti i bersagli devono essere nei paraggi al momento del lancio. \mbox{Persistente.}
	\item \spell{Mano Interposta} una mano blocca 5d6pf di Danno di un singolo bersaglio avversario.
	\item \spell{Metamorfosi Funesta} \save{FOR} o trasforma permanentemente una creatura in un animale innocuo.

\vfill
\break

	\item \spell{Muro di Forza} un grande muro Invisibile è immune al Danno. Dura per per d6 minuti.
	\item \spell{Muro di Pietra} crea un grande muro di pietra (16pf, Armatura~8) che può essere plasmato.
	\item \spell{Nube Mortale} puoi spostare lentamente sul terreno questa piccola nuvola. Le creature viventi che restano al suo interno subiscono una Perdita di d6 FOR se falliscono un \save{FOR}.
	\item \spell{Occhi Indiscreti} d6 occhi fluttanti vanno in perlustrazione per conto tuo.
	\item \spell{Passa Pareti} per la durata dell'Incantesimo, crea un passaggio attraverso un muro di legno o di pietra.
	\item \spell{Permutazione} un bersaglio consenziente Perde \mbox{da d4 a d12} Punti Abilità e un altro bersaglio ne recupera un pari ammontare a una qualsiasi Abilità. La scelta del dado, delle \mbox{Abilità} e dei bersagli spetta a te.
	\item \spell{Pietrificazione} \save{FOR} o il bersaglio è trasformato permanentemente in una statua.
	\item \spell{Portale Planare} apri un portale verso un'altra realtà. Funziona in entrambe le direzioni.
	\item \spell{Santuario Privato del Mistico} crea un'illusione che impedisce a chiunque di visualizzare o scrutare un'area. Persistente.
	\item \spell{Scrigno Segreto} nasconde uno scrigno prezioso in uno spazio extraplanare; puoi recuperarlo a volontà.
	\item \spell{Simbolo di Sonno} incidi una Runa che pone chi la legge in un sonno magico se fallisce un \save{VOL}. Il sonno magico dura quanto l'Incantesimo stesso. Persistente.
	\item \spell{Spezza Incantamento} al tuo tocco, liberi da incantamenti, trasformazioni, maledizioni e pietrificazioni.
	\item \spell{Suggestione di Massa} \save{VOL} o le creature sono costrette a seguire il piano d'azione pronunciato. I TS contro suggestioni che sono nocive per i bersagli sono tirati con Vantaggio.
	\item \spell{Sventura} risucchia l'energia vitale da d12 bersagli viventi infliggendo loro d12 Danno ciascuno. Quando bersagli le piante, tira il d12 per il Danno due volte e prendi il risultato maggiore.
	\item \spell{Tela Sanguinosa} riempie un'area grande definita a tuo piacimento con una densa ragnatela di fili affilati come rasoi. Chiunque provi a muoversi o ad agire al suo interno deve superare un \save{DES} o prende d10 Danno e pone fine immediatamente al proprio turno.
	\item \spell{Telecinesi} sposta un oggetto, attacca una creatura o scaglia via un oggetto o una creatura. Il Danno \mbox{dipende} dalla taglia del bersaglio dell'oggetto; le creature scagliate prendono l'appropriato Danno da caduta.
	\item \spell{Teletrasporto} ti trasporta istantaneamente in una località conosciuta distante non più di 100 miglia.
	\item \spell{Trasformazione della Terra} fango in roccia o roccia in fango.
\end{enumerate}

\vfill
\break
\end{document}
