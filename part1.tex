\documentclass[itdr]{subfiles}

\begin{document}

\chapter{Personaggi}
\label{ch:personaggi}
\index{Personaggi}
%\resetHeaders

\section{Tirare un Personaggio}
\index{Punteggi di Abilità}

Tira 3d6 per ciascun Punteggio di Abilità.

\index{Forza}
\index{FOR|see {Forza}}
\textbf{FOR}ZA --- tempra e prestanza fisica.

\index{Destrezza}
\index{DES|see {Destrezza}}
\textbf{DES}TREZZA --- furtività, atletica e riflessi.

\index{Volontà}
\index{VOL|see {Volontà}}
\textbf{VOL}ONTÀ --- autocontrollo e magia.

\index{Denaro!iniziale}
Poi, tira 3d6 come Tiro Extra. Se vuoi, puoi scambiare tra loro due di questi quattro punteggi per una volta. Il~Tiro Extra rappresenta il tuo denaro iniziale in Scellini.

Un Punteggio di Abilità di 10 è la media umana.

\index{Punti Ferita}
\index{PF|see {Punti Ferita}}
Tira d6 per i tuoi Punti Ferita, una misura di quanto sei al sicuro dal subire Danno Critico, il quale può mettere a repentaglio la tua stessa vita. Più PF significa che il personaggio è più al sicuro.

Infine, scegli uno singolo Tratto, un Background e~acquista un po’ di Equipaggiamento.

\vfill
\section{Tratti}
\index{Tratti}

Ora scegli un singolo Tratto e rifallo ogni volta che ottieni un nuovo Livello d’Esperienza.
\vfill
\feat{Assassino}
I tuoi attacchi Potenziati contro opponenti inconsapevoli o inermi bypassano i PF.

\vfill
\feat{Belva}\featmt
Puoi controllare un Compagno Animale in più. I tuoi Compagni Animali agiscono come un singolo branco ai tuoi ordini. Se un tuo Compagno Animale deve fare un Tiro Salvezza di VOL, potresti farlo tu al suo posto.

\vfill
\index{Armatura}
\feat{Blindato}\feathp
L’Armatura Completa non ti impone i suoi Svantaggi e~puoi usare uno scudo mentre la indossi.

\vfill
\feat{Berserker}\feathp
Dopo aver preso Danno in combattimento, Incrementi il dado del Danno della tua arma da mischia e ottieni Vantaggio al prossimo Tiro Salvezza contro Danno Critico fino alla fine del tuo prossimo turno.

\vfill
\feat{Cecchino}
Dopo aver fatto un attacco a distanza, ottieni un dado del Danno con la stessa arma e per lo stesso bersaglio finché non ne attacchi un altro o il combattimento non sia terminato.

\vfill
\feat{Combattente}\featmt\feathp
Ottieni un d4 come dado bonus del Danno.
\featadv{il dado bonus è Incrementato di uno.}

\break

\vfill
\feat{Comandante}
Una volta per combattimento, impartisci un comando, che non conta come tua azione, a un bersaglio alleato per Potenziarne l’attacco in questo turno o per fargli recuperare d6pf.

\vfill
\feat{Duellante}\feathp
Una volta per combattimento, e fino al suo termine, puoi concentrarti nel combattere un singolo bersaglio avversario adiacente, Potenziando i tuoi attacchi in mischia contro di esso e Compromettendo i suoi attacchi in mischia contro di te. Gli attacchi contro di te di ogni restante \mbox{opponente} sono Potenziati.

\vfill
\index{Guarigione}
\feat{Guaritore}
Ottieni medicamenti del valore di 5 Scellini.
Durante un Riposo, puoi consumare 5 Scellini di medicamenti per far recuperare d6 Punti Abilità Persi a te o a un bersaglio alleato o per avere 4 su 6 possibilità di rimuovere un problema di salute.

I medicamenti possono essere acquistati nella maggior parte degli insediamenti e sono utilizzabili soltanto da personaggi Guaritori.
\vfill
\index{Runico}
\index{Tomo}
\index{Incantesimi}
\index{Incantesimi!distintivo}
\index{Magia}
\feat{Mistico}\featmt
Puoi leggere il Runico e lanciare Incantesimi.

Ottieni il Focus e il Tomo del Mistico contenente le istruzioni per due Trucchetti e sei Incantesimi del 1° Cerchio. Rendi Distintivo uno di questi sei Incantesimi (vedi \textbf{\fullref{ch:magia}}).
\featadv{aggiungi al Tomo un Trucchetto e tre Incantesimi (di un Cerchio pari o inferiore al tuo Livello da Mistico). Rendi Distintivo un Incantesimo in più.}

\vfill
\index{Competenza}
\feat{Provetto}\featmt
Ottieni Vantaggio ai Tiri Salvezza relativi a due settori di \mbox{Competenza}: accudimento degli animali, atletica, furto con scasso, raggiro, tolleranza degli alcolici, rapidità, furtività, navigazione, negoziazione, seguire tracce, ecc. Quando non sei sotto pressione, non hai alcuna necessità di dovere fare quei Tiri Salvezza.

\vfill
\feat{Rissaiolo}\featmt\feathp
Quando non indossi un'armatura, hai un punteggio d’Armatura pari a 1. Ottieni un d4 come dado bonus per il Danno senz’armi.
\featadv{il dado bonus è Incrementato di uno.}

\clearpage

\vfill
\feat{Spaccone}\feathp
Mentre attacchi, puoi colpire un secondo bersaglio tirando il dado del Danno della tua arma senza dadi bonus.

\vfill
\index{Manovre}
\feat{Tattico}\feathp
Quando fai un attacco, potresti associarci una Manovra (spingere, sgambettare, disarmare, avvinghiarti al bersaglio per il suo prossimo turno, ecc.). L’attacco si svolge come al solito e l’opponente deve fare un Tiro Salvezza per evitare l’effetto aggiuntivo descritto da te.

\vfill
\index{Doni}
\feat{Taumaturgo}\featmt
Puoi usare la tua azione per rivelare strabilianti poteri. Scegli due Doni. Il tuo dado per i Doni è il d4.

Prima di manifestare un Dono, tira due dadi per i~Doni e sottrai il risultato più basso da quello più alto per determinare il tuo Potere (\textbf{P}). Se i risultati sono uguali, fallisci e non puoi manifestare alcun Dono prima di un Riposo, altrimenti il Riposo ti servirà per poter manifestare di nuovo lo stesso potere.

\begin{enumerate}
	\item \textbf{Armonia:} fino al tuo prossimo Riposo, \textbf{P} animali scelti da te non ti attaccano senza motivo e tu riesci a comprenderli.
	\item \textbf{Castigo:} sferri un colpo con \textbf{P} Danni bonus ignorando l’Armatura e le resistenze soprannaturali.
	\item \textbf{Comando:} pronunci una singola parola (avvicinati, fermati, scappa, ecc.) a cui \textbf{P} creature che falliscono il proprio Tiro Salvezza di VOL devono obbedire nel loro prossimo turno.
	\item \textbf{Controllo:} non crei, ma controlli per un minuto fiamme, saette o acque. Scagliate, fanno \textbf{P} Danni (Fuoco, Elettricità o Freddo) a un bersaglio.
	\item \textbf{Credenza:} ottieni \textbf{P/2} (arrotondando per eccesso) risposte sincere da un bersaglio rispondente. Non puoi rifarlo sullo stesso bersaglio per un giorno.
	\item \textbf{Egida:} riduci immediatamente di \textbf{P} il Danno che hai preso. Conta come tua azione per il tuo prossimo turno.
	\item \textbf{Legame:} un singolo animale si metterà al tuo servizio senza riserve per \textbf{P} giorni. Ripetere il Legame porrà fine a quello in corso.
	\item \textbf{Presagio:} prevedi l’esito (buono, cattivo, entrambe le cose o non chiaro) immediato di \textbf{P/2} (arrotondando per eccesso) azioni.
	\item \textbf{Risveglio:} al tuo tocco, un bersaglio che ha preso Danno \mbox{Critico}, ma che non è ancora morto, recupera \textbf{P} PF. Può agire al suo prossimo turno.	
	\item \textbf{Scacciata:} \textbf{P} creature contro natura che falliscono il proprio Tiro Salvezza su VOL sono respinte a meno che non vengano messe sotto attacco.
\end{enumerate}
\featadv{Scegli un Dono aggiuntivo.\\Il tuo dado per i Doni è Incrementato di uno.}

\break

\vfill
\index{Incantesimi!random}
\begin{dbox}
	\paragraph{Incantesimi Random e Selezione a Caso dei Doni (opzionale)}
	 Invece di \mbox{sceglierli} direttamente\safepageref{(vedi pagina }{incantesimi_random}{)}, tira per selezionare in modo casuale i tuoi Trucchetti e Incantesimi da Mistico o i tuoi Doni da Taumaturgo.
\end{dbox}

\vfill
\vfill
\dimage{features}{90pt}
\vfill
\vfill

\begin{dbox}
\subsection*{Creare i tuoi Tratti}

Puoi creare uno tuo Tratto e sottoporlo all’approvazione dall’Arbitro.

I Tratti orientati al combattimento di solito fanno tirare due volte i PF e prendere il risultato migliore.

Un tipico beneficio da Tratto potrebbe essere:
\begin{itemize}
	\item un dado del Danno bonus applicabile in talune circostanze (ai fini dell’Avanzamento, parte come un d4 e va Incrementato di uno)
	\item un effetto aggiuntivo per gli attacchi del personaggio
	\item un Vantaggio su di una specifica gamma di Tiri Salvezza
	\item un particolare stratagemma
\end{itemize}

Cercate di bilanciare i nuovi Tratti con quelli già esistenti.\hspace{-0.6em}
\end{dbox}

\vfill
\subsection*{Tratti e Lignaggi (opzionale)}
\index{Tratti!lignaggio}
\index{Lignaggio|see {Tratti, lignaggio}}

Se l’Arbitro consente l’uso di personaggi non umani (e unicamente durante la fase di creazione di un nuovo personaggio), puoi scegliere in alternativa uno dei Tratti in basso.

\vfill
\feata{Elfo}
Hai Vantaggio ai Tiri Salvezza contro la magia che altera la mente (ipnosi, sonno, stordimento, ecc.). Hai esperienza degli ambienti selvaggi e lì ti senti a tuo agio.

\vfill
\feata{Halfling}
Grazie alla tua piccola taglia, puoi nasconderti bene, riuscendo a infilarti in passaggi stretti e ad accedere in spazi ridotti. Puoi ripetere ogni tiro con cui hai fatto 20 e usare il nuovo risultato.

\feata{Nano}
Sei immune al veleno e puoi vedere al buio come se ci fosse della luce soffusa. Hai esperienza degli ambienti sotterranei e lì ti senti a tuo agio.

\clearpage

\vfill
\section{Background}
\index{Background}

Scegli la precedente carriera del tuo personaggio e pensa a una ragione che ci dica perché l’abbandonata per dedicarsi all'\mbox{Avventura}.

\vfill
\featb{Cacciatore}
Ottieni un’arma da guerra a distanza adatta alla caccia (arco lungo, moschetto semplice, ecc.) e una trappola per animali. Dimostri bravura nel cacciare e nel seguire tracce.

\vfill
\featb{Criminale}
Ottieni il tuo arnese preferito da criminale (manganello (sfollagente), piede di porco, rampino, grimaldelli, carte segnate o dadi truccati, ecc.), un pugnale e un contatto nel mondo del crimine.

\vfill
\featb{Marinaio}
Ottieni un compagno animale: un pappagallo parlante (FOR~6, VOL~6, 2pf, d4 Artigli), una scimietta (FOR~7, VOL~7, 3pf, d4 Morso), ecc. Ti intendi di navigazione marittima.

\vfill
\featb{Menestrello}
Ottieni uno strumento musicale. Grazie al tuo vasto repertorio, conosci un mucchio di leggende e di racconti e hai 4 possibilità su 6 di ricordare qualcosa di rilevante da esse.

\vfill
\featb{Nobile}
Raddoppi il tuo denaro iniziale. Il tuo nome ha ancora un certo peso.

\vfill
\featb{Operaio}
Ottieni esperienza in un tipo di lavoro (agricoltura, giardinaggio, pastorizia, taglio di legname, opere in muratura, estrazioni in miniera, ecc.), un’arma semplice da mischia appropriata, un paio di attrezzi, una \mbox{corda lunga 20 piedi} e 2d4 Scellini di compenso per il tuo ultimo lavoro. La gente \mbox{comune} ti considera sua pari.

\vfill
\featb{Soldato}
Ottieni un’arma da guerra e un rango militare.

\vfill
\featb{Studioso}
Ottieni un set da scrittura, un diario con i tuoi appunti e un libro sull’oggetto della tua specializzazione.

Hai 4 possibilità su 6 di conoscere un fatto all’interno della tua area di studio e ogni cosa relativa alla tua specializzazione (p. es. Storia (Archeologia)).

\vspace{5ex}
\break

\vfill
\begin{dbox}
\subsection*{Creare i tuoi Background}

Puoi creare un tuo Background e sottoporlo all’approvazione dell’Arbitro.

Di solito, il Background dovrebbe fornire oggetti (del valore grossomodo di 10--12 Scellini) dalla vita passata del personaggio e un qualche beneficio relativo all’interpretazione dello stesso personaggio.
\end{dbox}

\vfill
\dimage{backgrounds}{490pt}

\vfill
\clearpage


\section{Equipaggiamento}
\index{Equipaggiamento}
\index{Denaro}
\index{Penny|see {Denaro}}
\index{Scellino|see {Denaro}}
\index{Fiorino|see {Denaro}}

Dieci \textbf{Pennies} (p) fanno uno \textbf{Scellino} (s) e cento Scellini fanno un \textbf{Fiorino} (f).

\vfill

Tutti i personaggi trasportano dell'\textbf{equipaggiamento standard}, che comprende vestiti semplici, uno zaino, attrezzatura essenziale per accamparsi, sei torce e razioni per tre giorni.

\vfill

\index{Attacchi!senz'armi}
Gli \textbf{attacchi senz’armi} fanno d4 Danno.

\vfill

\index{Armi}
\index{Danno!arma}
Se sono indicati due dadi, il primo è per le armi a~\textbf{una mano} (1-mano), il secondo per quelle a \textbf{due mani} \mbox{(2-mani).}

\vfill

\index{Armi da Fuoco|see {Armi, armi da fuoco}}
Le \textbf{armi da fuoco} sono molto rumorose e ignorano l’Armatura. Ricaricare le armi da fuoco in combattimento richiede entrambe le mani e un turno intero stando fermi sul posto.

\vfill

\index{Rivendita}
Potresti provare a \textbf{rivendere} un oggetto alla metà del suo prezzo.

\vfill

\paragraphsection{Armi da Mischia:}

\weapon{Arma Semplice da Mischia}{1s}{d6}\\Solo a due mani. Attrezzi o armi la cui costruzione non è stata pensata per l’uso frequente in battaglia. Forcone, Bastone Ferrato, Martello da Fabbro, Maglio Spaccalegna, etc.

\weapon{Arma da Guerra da Mischia}{10s}{d6/d8}\\Armi di base, fatte apposta per l’uso in battaglia. Ascia, Pugnale, Alabarda, Mazza, Lancia, Spada, ecc.

\weapon{Arma Superiore da Mischia}{1f}{d8/d10}\\Armi di fattura sofisticata o di lavorazione magistrale.

\weapon{Lancia Lunga}{10s}{d8}\\In groppa a una cavalcatura può essere usata con uno scudo; a piedi può essere usata solo a due mani.
\vfill

\paragraphsection{Armi a Distanza:}

\weapon{Arma Semplice a Distanza}{1s}{d4}\\Attrezzi o armi la cui costruzione non è stata pensata per l’uso frequente in battaglia. Forcone. Freccette, Arco da Caccia, Fionda, Pugnali da Lancio, ecc.

\weapon{Arma da Guerra a Distanza}{10s}{d6}\\Armi di base, fatte apposta per l’uso in battaglia. \mbox{Balestra,} Arco Lungo, Moschetto Semplice o Rivoltella, ecc.

\weapon{Arma Superiore a Distanza}{1f}{d8}\\Archi, balestre e pistole pesanti o di fattura sofisticata. 

\vfill

\paragraphsection{Armature:}
\index{Armatura}
\index{Scudo}

\armour{Armatura Leggera}{10s}{1}

\armour{Armatura Completa}{1f}{2}\\Rende molto difficile correre, nuotare, nascondersi, ecc., imponendo Svantaggio sui Tiri Salvezza appropriati.

\equip{Scudo}{5s}: +1 Armatura.\\Si può usare solo un'arma e non ha effetto assieme all’Armatura Completa.

\break

\index{Oggetti}
\paragraphsection{Altri Oggetti:}

\index{Acido}
\equip{Acido}{10s a fiala}: a un singolo bersaglio, d4 Danno da Acido ora e, se non è lavato via, d4~Punti di FOR Persi (soggetti all'Armatura) alla fine del prossimo turno.

\index{Olio Incendiario}
\equip{Olio Incendiario}{10s a fiasca}: incendia un'area. Chiunque si trovi al suo interno prende d6 Danno da Fuoco ora e, se le fiamme non vengono spente, d6 alla fine del prossimo turno.

\index{Polvere Nera}
\equip{Polvere Nera}{20s a vaso}: va accesa con una miccia o una fiamma diretta. Chiunque si trovi nell'area prende d10 Danno da Scoppio.

\index{Attrezzatura d'Avventura}
\index{Attrezzatura|see {Attrezzatura d'Avventura}}
\index{Razioni}
\index{Cibo|see {Razioni}}
\index{Luce}
\equip{Attrezzatura d'Avventura}{5p cad.}: Corda di 10 piedi, Triboli (rallentano chi ti insegue), Gessetto, Dadi, Esca e Pietra Focaia, Razione di Cibo, Olio per Lampada, Cartapecora, Picchetto, Tenda, 6 Torce.

\index{Attrezzi}
\equip{Attrezzi}{1s cad.}: Trappola per Animali, Pertica Ripieghevole, Piede di Porco, Trapano, Canna da Pesca, Rampino, Accetta, Grimaldelli, Piccozza, Pala, Set da Scrittura.

\equip{Oggetti Costosi}{10s cad.}: Gioco da Tavolo, Libro, Vestiti Ricercati, Lanterna, Specchio, Clessidra, Cannocchiale.

\vfill

\paragraphsection{Varie ed Eventuali:}
\index{Veicoli!su acqua}
\index{Imbarcazioni}
\index{Barca a Remi}
\index{Galea}
\subparagraph{Imbarcazioni}: dalla Barca a Remi (50s) alla Galea (200f).

\index{Veicoli}
\index{Carretto}
\index{Carrozzone}
\subparagraph{Carri}: dal Carretto (30s) al Carrozzone (1f).

\index{Taverne}
\subparagraph{Taverne}: Pasto, Bevanda e Letto in un Postaccio (1p), in un Posto Niente Male (1s) o in un Posto di Classe (20s).

\index{Guarigione}
\index{Interventi Curativi}
\index{Perdita di Punti Abilità}
\index{Punti Abilità Persi|see {Perdita di Punti Abilità}}
\equip{Intervento Curativo}{10s}: ripristina nottetempo un Punto Abilità Perso o rimuove un altro problema di salute.

\index{Proprietà}
\index{Strutture}
\subparagraph{Proprietà}: Cottage (1f), Bottega (10f), Maniero (100f).

\index{Cavalcature}
\index{Mulo}
\index{Cavallo}
\subparagraph{Cavalli}: dal Mulo (20s) (FOR~14, VOL~5, 3pf) al~Cavallo~(1f) (FOR~16, DES~12, VOL~5, 3pf).

\index{Compagni Animali}
\index{Cani}
\index{Meticcio}
\index{Segugio}
\subparagraph{Cani}: dal Meticcio (5s) (FOR~8, VOL~6, 2pf, d4 Morso) al Segugio (50s) (5pf, d6 Morso).

\index{Uccelli}
\index{Pappagallo}
\index{Falco}
\subparagraph{Uccelli}: dal Pappagallo (5s) (FOR~6, VOL~6, 2pf, d4 Artigli) al Falco (50s) (FOR~8, VOL~8, 5pf, d6 Artigli).

\vfill

\index{Aiutanti}
\index{Tedoforo|see {Aiutanti}}
\index{Guida|see {Aiutanti}}
\index{Armigero|see {Aiutanti}}
\index{Specialista|see {Aiutanti}}
\index{Difensore|see {Aiutanti}}
\paragraphsection{Aiutanti:} (costo al giorno; d6pf, Punteggi di Abilità a 10
se non diversamente indicato)

\begin{itemize}
	\item Tedoforo (1s): VOL~8.
	\item Guida (2s): FOR~8, bastone (d6, 2-mani), lanterna, corda.
	\item Armigero (5s): FOR~12, Armatura~2 (leggera + scudo), lancia (d6).
	\item Specialista (10s): pugnale (d6), arco (d4), area di specializzazione.
	\item Difensore (50s): FOR~14, 6+d6pf, Armatura~2 (completa), alabarda (d8+d6, 2-mani), Combattente Riconosciuto.
\end{itemize}

%\break

%\dimagepage{equipment2}

\end{document}
